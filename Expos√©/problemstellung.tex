%% Einleitung.tex
%% $Id: einleitung.tex 28 2007-01-18 16:31:32Z bless $
%%

\section{Problemstellung}
\label{ch:Problemstellung}
%% ==============================
% CLEARLY SHOW CONTRIBUTIONS AND LINK THEM TO SECTIONS

%- Mit welchem Bereich beschäftigen wir uns?
%- Welche Probleme treten hier auf?
%- Wie werden diese bisher gelöst?
%- Was wollen wir beitragen?

Deutschland weißt seit 1998 einen Anstieg von Krankheitstagen, bedingt durch psychische Diagnosen, auf. Gleichzeitig wird ein Rückgang von Krankmeldungen durch andere Diagnosen registriert.(vgl.\cite{Jacobi2014}, S. 85) Psychische und lebensstil assoziierte Erkrankungen haben in den letzten Jahrzehnten demnach deutlich an Relevanz zugenommen. Diese Entwicklung ist gekoppelt mit einem höheren Bedarf an psychotherapeutischen Behandlungsplätzen, der zur Folge hat, dass die Wartezeiten auf eine erste Psychotherapeutische Sprechstunde im Schnitt 5,7 Wochen beträgt (vgl.\cite{Microsof77:online}, S.5). Auch Krankenkassen beobachten den Anstieg der psychischen Diagnosen. So betont Prof. Dr. Christoph Straub, Vorstandsvorsitzen der BARMER im Zuge des Ärztereports 2018 
\begin{quote}
„Vieles spricht dafür, dass es künftig noch deutlich mehr psychisch kranke junge Menschen geben wird. Gerade bei den angehenden Akademikern steigen Zeit- und Leistungsdruck kontinuierlich, hinzu kommen finanzielle Sorgen und Zukunftsängste. Vor allem mehr niedrigschwellige Angebote können helfen, psychische Erkrankungen von vorn herein zu verhindern“ (vgl.\cite{Arztrepo90:online})
\end{quote}
Auch auf dem Arbeitsmarkt und in der Wirtschaft macht sich diese Entwicklung bemerkbar. Neben einer größeren Anzahl an Ausfalltagen bei Menschen mit psychischer Erkrankung, ist die Dauer der Krankschreibung ebenfalls erhöht.(vgl.\cite{Nubling2014}, S.393) Ebenso haben Berentungen aufgrund einer psychischen Erkrankung zugenommen(vgl.\cite{Jacobi2014}\cite{Nubling2014}, S.77). Für das Jahr 2008 wurde durch psychische und Verhaltensstörungen ein Produktionsausfall von 26 Mrd. Euro und ein Ausfall an Bruttowertschöpfung von 45 Mrd. Euro (1,8 Prozent des Bruttoinlandsprodukt) ermittelt. (vgl.\cite{EntwurfeinesDreizehntenGesetzeszurAnderungdesAtomgesetzesBundesregierungDeutschland2012}, S.12)


%- Mehr und mehr psychisch Erkrankte, tendenz steigend
%- Bedarf an Therapieplätzen wird nicht gedeckt -> hohe Wartezeiten
%- In manchen Bundesländern bis zu <Zahl einfügen> Wochen auf Therapieplatz warten
%- Oft wird eine Begleitung durch den Tag benötigt
%- Es hat sich gezeigt, dass Menschen zu Chatbots eine Bindung entwickeln
%- Diese Bindung soll zur Therapy genutzt werden (Beispielprojekte Vor- und Nachteile)
%- Movisens will hier ansetzen und einen Therapybuilder entwickeln
%- Auf Nachteil so und so gehe ich mit meiner Arbeit ein - Ziel so und so wird im Rahmen dieser Thesis bearbeitet
%- TML entwickelt die es psychologen ermöglicht ohne programmierkenntnisse Therapien anzulegen
%- diese sollen vom Patient in Form eines Chatbots bearbeitet werden