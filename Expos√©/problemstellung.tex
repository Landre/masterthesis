%% Einleitung.tex
%% $Id: einleitung.tex 28 2007-01-18 16:31:32Z bless $
%%

\section{Problemstellung}
\label{ch:Problemstellung}
%% ==============================
% CLEARLY SHOW CONTRIBUTIONS AND LINK THEM TO SECTIONS

%- Mit welchem Bereich beschäftigen wir uns?
%- Welche Probleme treten hier auf?
%- Wie werden diese bisher gelöst?
%- Was wollen wir beitragen?

Sie geben Auskunft über das Wetter (vgl. \cite{GoogleAl38:online}), nehmen Bestellungen entgegen (vgl. \cite{KassenSc50:online}) oder wirken als Coach (vgl. \cite{Wysayour57:online}) - Chatbots werden bereits vielseitig im Alltag eingesetzt. Auch die Psychologie profitiert von diesen Entwicklungen. 1966 entwickelte Joseph Weizenbaum mit \emph{ELIZA} den ersten Chatbot. \emph{ELIZA} sollte seinem menschlichen Gesprächspartner das Gefühl geben, dass dieser mit einem Psychiater über eine Chatoberfläche kommuniziert. Entwickelt wurde \emph{ELIZA} allerdings nicht mit der Absicht  Psychotherapie zugänglich zu machen, sondern um ein Modell zur maschinellen Verarbeitung von natürlicher Sprache zu implementieren (vgl. \cite{Weizenbaum1966}). Was mit Joseph Weizenbaums \emph{ELIZA} begann, brachte mit der Entwicklung der Forschung und Technik schließlich einige Chatbots, wie \emph{Wysa} (vgl. \cite{Wysayour57:online}), \emph{Woebot} (vgl. \cite{WoebotYo93:online}) und \emph{Tess} (vgl. \cite{TessArti99:online}), im Bereich der psychischen Gesundheit hervor. Sie stellen heutzutage verschiedene Methoden der kognitiven Verhaltenstherapie bereit, die Nutzern helfen können, deren Introspektion zu verbessern. Dabei wirken sie wie ein Coach der jederzeit erreichbar ist (vgl. \cite{Fitzpatrick2017}).  

In den 1960-ern hatten nur wenige Zugang zu Computern. Durch ihre Bauweise benötigten diese nicht nur viel Platz, sie waren zu dieser Zeit auch sehr kostspielig (vgl. \cite{SWB-11524946X}). Die Technik hat sich allerdings über die Jahrzehnte hinweg stark verändert. Nicht nur wurden Computer erschwinglich und haben eine deutlich größere Rechenleistung, sie begleiten uns mittlerweile auch in Form eines Tablets oder Laptops als Personal Computer durch den Alltag. Seit Apple ihr erstes Smartphone \emph{Iphone} im Jahr 2007 einführte, eröffneten sich durch diese Geräte noch weitere technische Möglichkeiten. Smartphones entwickelten sich zu kleinen, handlichen Geräten die nahezu in jeder Tasche Platz finden (vgl. \cite{SWB-481290869}). Außerdem beinhalten die Geräte heutzutage verschiedene Sensoren, haben Zugriff auf eine Vielzahl von Anwendungen und können sich mit dem Internet verbinden (vgl. \cite{SWB-481290869}\cite{AppStore21:online}). Die Handlichkeit und Vielzahl an mitgebrachten Funktionen führte dazu, dass im Jahre 2018 allein in Deutschland 22,74 Millionen Smartphones verkauft wurden (vgl. \cite{Zukunftd37:online}). Statistiken der \emph{Bitkom Research} ermittelten, dass im Jahr 2017 78 Prozent der Deutschen ein Smartphone verwendeten (vgl. \cite{Smartpho6:online}).

Entwickler nutzen die technischen Vorteile der Smartphones und Personal Computer. So begleitet \emph{Woebot} Menschen mit Depressionen oder inneren Unruhen mit Techniken aus der kognitiven Verhaltenstherapie als Selbsthilfe durch den Alltag (vgl. \cite{Fitzpatrick2017}). Der Nutzer kann dabei auswählen, ob dieser über eine \emph{Iphone}-App, \emph{Android}-App oder via \emph{Facebook Messenger} mit \emph{Woebot} kommunizieren möchte (vgl. \cite{WoebotYo93:online}). Letzteres ist auf jedem browserfähigen Gerät nutzbar. 

%Eine Studie der \emph{Stanford School of Medicine} untersuchte den Einsatz von \emph{Woebot} hinsichtlich seiner Realisierbarkeit, Nutzerakzeptanz und die vorläufige Wirksamkeit des bereitgestellten Selbsthilfeprogramms. Das Ergebnis der Studie zeigte, dass \emph{Woebot} beinahe täglich von 31 Probanden genutzt wurde. Außerdem ließ sich bei diesen ein positiver Einfluss hinsichtlich ihrer Depressionsbewältigung und dem Umgang mit inneren Unruhen messen (vgl. \cite{Fitzpatrick2017}). 

Eine Studie der Stanford School of Medicine untersuchte den Einsatz des Chatbots \emph{Woebot} hinsichtlich seiner Realisierbarkeit, Nutzerakzeptanz und die vorläufige Wirksamkeit des bereitgestellten Selbsthilfeprogramms. Verglichen wurden dabei zwei Gruppen. Eine dieser Gruppen, bestehend aus 31 Probanden, erhielt Zugriff auf \emph{Woebot}. Die zweite Gruppe, bestehend aus 25 Probanden, erhielt Zugriff auf das Ebook \emph{Depression} des \emph{National Institute of Mental Health}. Die Studiendauer wurde auf zwei Wochen festgelegt. Nach Ablauf der Studie zeigte sich, dass die Mehrheit der \emph{Woebot}-Gruppe beinhahe täglich den Chatbot nutzte. Auch konnte bei der Nutzung des \emph{Woebots} im Vergleich zur Nutzung des Ebooks eine größere Zufriedenheit festgestellt werden. Außerdem ließ sich bei dieser Gruppe ein signifikanten, positiven Einfluss hinsichtlich ihrer Depressionsbewältigung und dem Umgang mit inneren Unruhen messen (vgl. \cite{Fitzpatrick2017}).

Eine weitere Studie testete den Einfluss eines virtuellen Akteurs auf das Nutzerverhalten innerhalb eines klinischen Interviews. In dieser Studie wurden 145 Probanden in zwei Gruppen eingeteilt. 57 dieser Probanden führten einen Dialog mit einem virtuellen Akteur, der von einem Menschen gesteuert wurde. Die restlichen 88 Probanden unterhielten sich mit einem virtuellen Akteur, der mittels künstlicher Intelligenz kommunizierte. Das jeweilige Setting der Gruppen war allen Probanden bekannt. Gemessen wurde, unter anderem anhand eines Fragebogens, die Furcht vor negativer Bewertung (FNE), das Selbstdarstellungsverhalten (IM), die Nutzbarkeit des Systems (SU) sowie die Selbsttäuschung der Probanden (SD). Die Ergebnisse zeigten auf, dass signifikante Unterschiede zwischen den Gruppen gemessen werden konnte. So wurde festgestellt, dass Probanden, die Dialoge mit der künstlichen Intelligenz führte, einen niedrigeren FNE und IM Wert aufweisen (vgl. \cite{Gratch2014}). 

%Die Erkenntnis, dass Chatbots, wie \emph{Woebot}, einen positiven Einfluss auf die Depressionsbewältigung haben können, zeigt auf, dass Chatbots im Bereich der Psychologie 

Diese Ergebnisse zeigen auf, dass Chatbots im Bereich der Psychologie und Psychotherapie nützliche Werkzeuge sein können. Allerdings ist das Entwickeln solcher Chatbots für Psychologen noch immer eine Hürde. Zwar gibt es zahlreiche Baukästen zur Entwicklung von Chatbots die keine tiefgreifenden Programmierkenntnisse benö-tigen. Diese sind jedoch überwiegend auf den Bereich des Marketings ausgerichtet, weshalb sie in ihrem Funktionsumfang meist eingeschränkt sind. Baukästen die mehr Funktionalität bieten, benötigen lange Einarbeitungszeit und Expertenwissen in Bezug auf ihre Programmierung. Eine einfache und schnelle Umsetzung ist daher oft nicht möglich. Auch die Entwicklung eines eigenen Produktes birgt für Psychologen und Softwareunternehmen Probleme. So scheitert die Umsetzung unter anderem an Kommunikationshürden zwischen Entwicklern und Psychologen. Aber auch die komplexen Anforderungen des Medizinproduktegesetztes (MPG), die medizinische Produkte für die Herstellung oder Einführung in den Europäischen Wirtschaftsraum zu erfüllen haben, stellen eine Hürde dar (vgl. \cite{MPGnicht8:online}).  

Das Unternehmen \emph{movisens GmbH} entwickelt derzeit das Projekt \emph{TherapyBuilder} welches Psychologen und Psychotherapeuten die Möglichkeit bieten soll, Chatbots  für Studien sowie zur Therapiebegleitung einzusetzen. Im Rahmen dieser Masterarbeit wird für das Projekt \emph{TherapyBuilder} ein Modellierungsansatz \emph{TMA} (Therapy Modelling Approach) entwickelt. Ziel dieses \emph{TMA} ist es, Psychologen die Autonomie zu geben, ohne Expertenwissen Chatbots zu erstellen, um diese in Studien und therapiebegleitend einzusetzen.



%- ca. 250 Apps im Google Play Store unter dem Suchbegriff `Psychische Gesundheit''
%- ca. 255 Apps im Google Play Store unter dem Suchbegriff `mental health''

 
%- Es hat sich gezeigt, dass Menschen zu Chatbots eine Bindung entwickeln
%- Diese Bindung soll zur Therapy genutzt werden (Beispielprojekte Vor- und Nachteile)
%- Movisens will hier ansetzen und einen Therapybuilder entwickeln
%- Auf Nachteil so und so gehe ich mit meiner Arbeit ein - Ziel so und so wird im Rahmen dieser Thesis bearbeitet
%- TML entwickelt die es psychologen ermöglicht ohne programmierkenntnisse Therapien anzulegen
%- diese sollen vom Patient in Form eines Chatbots bearbeitet werden