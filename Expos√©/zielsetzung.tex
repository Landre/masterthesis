%% grundlagen.tex
%% $Id: grundlagen.tex 28 2007-01-18 16:31:32Z bless $
%%

\section{Zielsetzung \& Erkenntnisinteresse}
\label{ch:Zielsetzung}

%Erklärung dessen, was du am Ende deines Forschungssprojekts herausgefunden haben willst und weshalb

Ziel der Arbeit ist die Konzeption einer Therapiemodellierungsansatz (\emph{TMA}). Diese soll es erlauben, auch technisch wenig versierten Psychologen ihre Therapieideen in einer Art und Weise zu formulieren, die von einer Maschine verarbeitet und ausgeführt werden kann. Dadurch entfällt der hohe und fehleranfällige Abstimmungsaufwand zwischen Forschern und Entwicklern. 

Durch den Einsatz der \emph{TMA} sollen \emph{MPG} konforme Anwendungen mit einem Chatbot UI entstehen, welche eine für den Patienten vertraute, dialogähnliche Kommunikation ermöglicht. Dies erlaubt es eine stärkere persönliche Bindung zwischen App und dem Patienten herzustellen, was den Therapieerfolg unterstützen soll. 

In der Arbeit gilt es vor allem die komplexen Konfigurationsmöglichkeiten der Domäne einer digitalen Therapie funktional abzubilden. Durch eine Befragung der Anwendergruppen soll Domänenwissen erarbeitet werden und im Folgenden die \emph{TMA} und dessen grafische Repräsentation iterativ entworfen werden. Dabei gilt ein hohes Augenmerk der Usability, um sicherzustellen, dass der Aufwand zur Therapieentwicklung und Studiendurchführung nicht größer ist, als derzeitige Methoden zur Therapieentwicklung und Studiendurchführung.

Es werden zwei Modellierungsansätze entwickelt. Einer dieser Ansätze wird in Form eines komplexen Mockups umgesetzt. Ein weiterer in Form eines Prototypen, welcher aus der Anpassung eines Experience Sampling Tools resultiert. Anschließend erfolgt eine Evaluation der Entwürfe in einer Vergleichsstudie.
