%% grundlagen.tex
%% $Id: grundlagen.tex 28 2007-01-18 16:31:32Z bless $
%%

\section{Zielsetzung \& Erkenntnisinteresse}
\label{ch:Zielsetzung}

%Erklärung dessen, was du am Ende deines Forschungssprojekts herausgefunden haben willst und weshalb

Ziel der Arbeit ist die Konzeption und Entwicklung einer Therapiemodellierungssprache. Diese soll es erlauben, technisch wenig versierten Forschern ihre Therapieideen in einer Art und Weise zu formulieren, die eine Maschine verstehen und ausführen kann. Dadurch entfällt der hohe und fehleranfällige Abstimmungsaufwand zwischen Forschern und Entwicklern. Durch den Einsatz der Therapiemodellierungssprache sollen MPG konforme Anwendungen mit einer conversational UI entstehen, welche eine für den Patienten vertraute, dem Therapiegespräch ähnelnde Kommunikation ermöglicht. Dies erlaubt es eine stärkere persönliche Bindung zwischen App und dem Patienten herzustellen, was den Therapieerfolg unterstützen soll. 

In der Arbeit gilt es jedoch vor allem die komplexen Konfigurationsmöglichkeiten der Domäne einer digitalen Therapie funktional abzubilden. Durch eine Befragung der Anwendergruppen soll Domänenwissen erarbeitet werden und im Folgenden die Therapiemodellierungssprache und dessen grafische Repräsentation iterativ entworfen werden. Dabei gilt ein hohes Augenmerk der Usability und User Experience, die gerade in der Therapie-Domäne einen hohen Stellenwert einnehmen muss.

Diese entworfene Modellierungssprache soll prototypisch umgesetzt werden und in einer kleinen Usability-Studie evaluiert werden.
