%% Einleitung.tex
%% $Id: einleitung.tex 28 2007-01-18 16:31:32Z bless $
%%

\section{Problemstellung}
\label{ch:Problemstellung}
%% ==============================
% CLEARLY SHOW CONTRIBUTIONS AND LINK THEM TO SECTIONS

%- Mit welchem Bereich beschäftigen wir uns?
%- Welche Probleme treten hier auf?
%- Wie werden diese bisher gelöst?
%- Was wollen wir beitragen?

Deutschland weißt seit 1998 einen Anstieg von Krankheitstagen, bedingt durch psychische Diagnosen, auf. Gleichzeitig wird ein Rückgang von Krankmeldungen durch andere Diagnosen registriert. (vgl.\cite{Jacobi2014}, S. 85) Psychische und Lebensstil assoziierte Erkrankungen haben in den letzten Jahrzehnten demnach deutlich an Relevanz zugenommen. Diese Entwicklung ist gekoppelt an einen höheren Bedarf an psychotherapeutischen Behandlungsplätze, welcher sich an den Wartezeiten für eine erste psychotherapeutische Sprechstunde bemerkbar macht. Diese liegen derzeit im Schnitt bei 5,7 Wochen (vgl.\cite{Microsof77:online}, S.5). Auch Krankenkassen beobachten den Anstieg der psychischen Diagnosen. So betont Prof. Dr. Christoph Straub, Vorstandsvorsitzen der BARMER im Zuge des Ärztereports 2018 
\begin{quote}
„Vieles spricht dafür, dass es künftig noch deutlich mehr psychisch kranke junge Menschen geben wird. Gerade bei den angehenden Akademikern steigen Zeit- und Leistungsdruck kontinuierlich, hinzu kommen finanzielle Sorgen und Zukunftsängste. Vor allem mehr niedrigschwellige Angebote können helfen, psychische Erkrankungen von vorn herein zu verhindern“ (vgl.\cite{Arztrepo90:online})
\end{quote}
Auf dem Arbeitsmarkt und in der Wirtschaft lassen sich ebenfalls Auswirkungen der psychischen Verhaltensstörungen erkennen. Neben einer größeren Anzahl an Ausfalltagen bei Menschen mit psychischer Erkrankung, ist die Dauer der Krankschreibung ebenfalls erhöht. (vgl.\cite{Nubling2014}, S.393) Ebenso haben Berentungen aufgrund einer psychischen Erkrankung zugenommen (vgl.\cite{Jacobi2014}\cite{Nubling2014}, S.77). Für das Jahr 2008 wurde durch psychische und Verhaltensstörungen ein Produktionsausfall von 26 Mrd. Euro und ein Ausfall an Bruttowertschöpfung von 45 Mrd. Euro (1,8 Prozent des Bruttoinlandsprodukt) ermittelt. (vgl.\cite{EntwurfeinesDreizehntenGesetzeszurAnderungdesAtomgesetzesBundesregierungDeutschland2012}, S.12)

Die hohen wirtschaftlichen Kosten, Wartezeiten auf Therapieplätze und steigenden Zahlen der psychischen Diagnosen deuten auf einen immensen Bedarf an wirksamen Therapiemethoden für psychische und Verhaltensstörungen hin. Auch haben die Wahrnehmung und Akzeptanz psychischer Probleme innerhalb der Gesellschaft deutlich zugenommen. Die Krankenkassen reagieren auf diese Tendenz mit einem Angebot aus Selbsthilfe Coachings für Depressionen und Burnout (vgl.\cite{Hilfebei71:online}\cite{TKDepres18:online}) oder online Trainings für psychische Beschwerden (vgl.\cite{PROMIND78:online}). Besonders in der App-Branche macht sich diese Entwicklung bemerkbar. Derzeit gibt es in einem Marktbestimmenden AppStore rund 250 Apps zum Thema \emph{Psychische Gesundheit}(vgl. \cite{psychisc90:online}). Darunter befinden sich Anwendungen, die Conversational Agents zur kognitiven Verhaltenstherapie einsetzen. Eine Studie der \emph{Stanford School of Medicine} untersuchte den Einsatz dieser hinsichtlich ihrer Realisierbarkeit, Nutzerakzeptanz und die vorläufige Wirksamkeit des bereitgestellten Selbsthilfeprogramms. Genutzt wurde während der Studie der Conversational Agent \emph{Woebot}. Das Ergebnis der Studie zeigte, dass \emph{Woebot} beinahe täglich von den Probanden genutzt wurde. Außerdem ließ sich bei diesen eine Verringerung der Depression feststellen. Die Ergebnisse zeigen auf, dass Conversational Agents, wie \emph{Woebot} eine potentielle Alternative zu üblichen Kognitiven Verhaltenstherapie bieten (vgl.\cite{Hany1997}, S. 1) 

Trotz des gestiegenen öffentlichen Interesses  gibt es kaum Anwendungen, die auch Therapeuten die Möglichkeit bieten die Vorteile eines Conversational Agents für Therapien zu nutzen. Oft bieten diese nicht die benötigten Funktionalitäten die für eine Therapie benötigt werden oder setzen zu viel technisches Wissen voraus. Auch die Entwicklung eines eigenen Produktes birgt für Psychotherapeuten und Softwareunternehmen Probleme. So scheitert die Umsetzung unter anderem an Kommunikationshürden zwischen Entwicklern und Psychotherapeuten oder den komplexen Anforderungen, die medizinische Produkte zu erfüllen haben. 

Das Unternehmen \emph{movisens GmbH} entwickelt derzeit das Projekt \emph{TherapyBuilder} welches Psychotherapeuten die Möglichkeit bieten soll, Conversational Agents für Theapien einzusetzen. Im Rahmen dieser Masterarbeit wird für das Projekt \emph{TherapyBuilder} die Modelliersungssprache \emph{TML} (Therapymodellinglanguage) entwickelt. Ziel dieser TML ist es, Psychotherapeuten die autonomie zu geben, ohne Programmierkenntnisse oder tiefergreifendes technisches Wissen, einen Chatbot zu erstellen um neue Therapiemethoden zu entwickeln.



%- ca. 250 Apps im Google Play Store unter dem Suchbegriff `Psychische Gesundheit''
%- ca. 255 Apps im Google Play Store unter dem Suchbegriff `mental health''


%- Es hat sich gezeigt, dass Menschen zu Chatbots eine Bindung entwickeln
%- Diese Bindung soll zur Therapy genutzt werden (Beispielprojekte Vor- und Nachteile)
%- Movisens will hier ansetzen und einen Therapybuilder entwickeln
%- Auf Nachteil so und so gehe ich mit meiner Arbeit ein - Ziel so und so wird im Rahmen dieser Thesis bearbeitet
%- TML entwickelt die es psychologen ermöglicht ohne programmierkenntnisse Therapien anzulegen
%- diese sollen vom Patient in Form eines Chatbots bearbeitet werden