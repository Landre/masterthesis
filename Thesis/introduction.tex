%% Einleitung.tex
%% $Id: einleitung.tex 28 2007-01-18 16:31:32Z bless $
%%

\chapter{Einleitung}
\label{ch:Introduction}
%% ==============================
% CLEARLY SHOW CONTRIBUTIONS AND LINK THEM TO SECTIONS

%Sie geben Auskunft über das Wetter, nehmen Bestellungen entgegen oder wirken als Coach für den Alltag - Chatbots werden bereits vielseitig eingesetzt. Seit der Entwicklung von ELIZA, dem ersten Computer Programm welches eine Kommunikation zwischen Mensch und Maschine in natürlicher Sprache herstellt \cite{Weizenbaum1966}, wurden die unterschiedlichsten Chatbots entwickelt. Zunächst wurden sie programmiert, um festzustellen, ob diese so umgesetzt werden können, dass ein Mensch nicht unterscheiden kann ob dieser eine Unterhaltung mit einem Menschen oder einer Maschine führt. Doch über die Jahre wurden Chatbots für neue Aufgabenbereiche entwickelt. Mittlerweile bedienen sie die unterschiedlichsten Zwecke. Sie automatisieren einfache Kundenanfragen, dienen als Suchmaschine oder bieten Unterstützung für Menschen mit Depressionen. Genutzt werden sie heutzutage überwiegend auf Smartphones oder Computern. Sie sind damit nicht nur jederzeit verfügbar, sondern haben außerdem die Möglichkeit auf verschiedene Sensoren zuzugreifen, die im Gerät bereits mitgeliefert werden. 



%Sie geben Auskunft über das Wetter, nehmen Bestellungen entgegen oder wirken als Coach für den Alltag - Chatbots werden bereits vielseitig eingesetzt. Was einst mit ELIZA als Antwort auf den von Alan Turing beschriebenen Turing-Test begann, hat sich zum praktischen Helfer im Alltag entwickelt. Während Siri Anrufe tätigt oder Suchanfragen bearbeitet, steuert Amazon Echo die Beleuchtung des Wohnraumes oder spielt Hörbücher ab - alles via Sprachsteuerung. Auch Firmen haben das Potential der Chatbots erkannt. Sie setzen sie ein um Kundenanfragen oder Kundenbestellungen automatisiert zu bearbeiten. Genutzt werden sie heutzutage überwiegend auf Smartphones oder Computern. Sie sind damit nicht nur jederzeit verfügbar, sondern haben außerdem die Möglichkeit auf verschiedene Sensoren zuzugreifen, die im Gerät bereits mitgeliefert werden. Kommuniziert wird mit dem Chatbot über eine Chatoberfläche oder via Sprachsteuerung. So gestaltet sich die Bedienung des Chatbots besonders einfach. 



%Die Entwickler des Chatbots \emph{Woebot} nutzen die technischen Vorteile die Smartphones und Computer mit sich bringen. So begleitet \emph{Woebot} Menschen mit Depressionen oder inneren Unruhen mit Techniken aus der Kognitiven Verhaltenstherapie als Selbsthilfe durch den Alltag. Eine Studie der \emph{Stanford School of Medicine} untersuchte den Einsatz von \emph{Woebot} hinsichtlich seiner Realisierbarkeit, Nutzerakzeptanz und die vorläufige Wirksamkeit des bereitgestellten Selbsthilfeprogramms. Das Ergebnis der Studie zeigte, dass \emph{Woebot} beinahe täglich von den Probanden genutzt wurde. Außerdem ließ sich bei diesen ein positiver Einfluss messen hinsichtlich ihrer Depressionsbewältigung und dem Umgang mit inneren Unruhen. 



Sie geben Auskunft über das Wetter, nehmen Bestellungen entgegen oder wirken als Coach - Chatbots werden bereits vielseitig im Alltag eingesetzt. Auch die Psychologie profitiert von diesen Entwicklungen. 1966 entwickelte Joseph Weizenbaum mit \emph{ELIZA} den ersten Chatbot. \emph{ELIZA} sollte seinem menschlichen Gesprächspartner das Gefühl geben, dass dieser mit einem Psychiater über eine Chatoberfläche kommuniziert. Entwickelt wurde \emph{ELIZA} allerdings nicht mit der Absicht  Psychotherapie zugänglich zu machen, sondern um ein Modell zur maschinellen Verarbeitung von natürlicher Sprache zu implementieren. Was mit Joseph Weizenbaums \emph{ELIZA} begann, brachte mit der Entwicklung der Forschung und Technik schließlich einige Chatbots, wie \emph{Wysa}, \emph{Woebot} und \emph{Tess}, im Bereich der psychischen Gesundheit hervor. Sie stellen heutzutage verschiedene Methoden der Kognitiven Verhaltenstherapie bereit, die Nutzern helfen können deren Introspektion zu verbessern und die Methoden direkt anzuwenden. Dabei wirken sie wie ein Coach der jederzeit erreichbar ist. 

In den 1960-ern hatten nur wenige Zugang zu Computern. Durch ihre Bauweise benötigten diese nicht nur viel Platz, sie waren zu dieser Zeit auch sehr kostspielig. Die Technik hat sich allerdings über die Jahrzehnte hinweg stark verändert. Nicht nur wurden Computer erschwinglich und haben eine deutlich größere Rechenleistung, sie begleiten uns mittlerweile auch in Form eines Tablets oder Laptops als Personal Computer durch den Alltag. Seit Apple ihr erstes Smartphone \emph{Iphone} im Jahr 2007 einführte, eröffneten sich durch diese Geräte noch weitere technische Möglichkeiten. Smartphones entwickelten sich zu kleinen, handlicen Geräten die nahezu in jeder Tasche Platz finden. Außerdem beinhalten die Geräte heutzutage verschiedene Sensoren, haben Zugriff auf eine Vielzahl von Anwendungen und können sich mit dem Internet verbinden. Die Handlichkeit und Vielzahl an mitgebrachten Funktionen führte dazu, dass Smartphones im Jahre 2018 allein in Deutschland von bis zu 57 Millionen Personen genutzt wurden. 

Chatbot-Entwickler nutzen die technischen Vorteile die Smartphones und Personal Computer mit sich bringen. So begleitet \emph{Woebot} Menschen mit Depressionen oder inneren Unruhen mit Techniken aus der Kognitiven Verhaltenstherapie als Selbsthilfe durch den Alltag. Der Nutzer kann dabei auswählen, ob dieser über eine \emph{Iphone}-App, \emph{Android}-App oder via \emph{Facebook Messenger} mit \emph{Woebot} kommunizieren möchte. Letzteres ist auf jedem browserfähigen Gerät nutzbar. Eine Studie der \emph{Stanford School of Medicine} untersuchte den Einsatz von \emph{Woebot} hinsichtlich seiner Realisierbarkeit, Nutzerakzeptanz und die vorläufige Wirksamkeit des bereitgestellten Selbsthilfeprogramms. Das Ergebnis der Studie zeigte, dass \emph{Woebot} beinahe täglich von den Probanden genutzt wurde. Außerdem ließ sich bei diesen ein positiver Einfluss hinsichtlich ihrer Depressionsbewältigung und dem Umgang mit inneren Unruhen messen.

%Die Erkenntnis, dass Chatbots, wie \emph{Woebot}, einen positiven Einfluss auf die Depressionsbewältigung haben können, zeigt auf, dass Chatbots im Bereich der Psychologie 

Diese Ergebnisse zeigen auf, dass Chatbots im Bereich der Psychologie und Psychotherapie nützliche Werkzeuge sein können. Allerdings ist das Entwickeln solcher Chatbots für Psychologen noch immer eine Hürde. Zwar gibt es zahlreiche Baukästen zur Entwicklung von Chatbots die keine tiefgreifenden Programmierkenntnisse benötigen. Diese sind jedoch überwiegend auf den Bereich des Marketings ausgerichtet, weshalb sie in ihrem Funktionsumfang meist eingeschränkt sind. Baukästen die mehr Funktionalität bieten, benötigen lange Einarbeitungszeit und Expertenwissen in Bezug auf ihre Programmierung. Eine einfache und schnelle Umsetzung ist daher oft nicht möglich. Auch die Entwicklung eines eigenen Produktes birgt für Psychologen und Softwareunternehmen Probleme. So scheitert die Umsetzung unter anderem an Kommunikationshürden zwischen Entwicklern und Psychologen oder den komplexen Anforderungen, die medizinische Produkte zu erfüllen haben. 

Das Unternehmen \emph{movisens GmbH} entwickelt derzeit das Projekt \emph{TherapyBuilder} welches Psychologen sowie Psychotherapeuten die Möglichkeit bieten soll, Chatbots  für Studien sowie zur Therapiebegleitung einzusetzen. Im Rahmen dieser Masterarbeit wird für das Projekt TherapyBuilder die Modelliersungssprache \emph{TML} (Therapy Modelling Language) entwickelt. Ziel dieser \emph{TML} ist es, Psychologen die autonomie zu geben, ohne Expertenwissen, Chatbots zu erstellen um diese in Studien und Therapiebegleitend einzusetzen.

%- Es gibt viele Chatbots die mittlerweile im Alltag genutzt werden
%- oft für automatische bearbeitung von Kundenanfragen
%- Es hat sich rausgestellt, dass Chatbots gerne eingesetzt und gerne von Kunden verwendet werden
%- Außerdem bieten sie viele Vorteile (ständig verfügbar, können auf verschiedene Daten zugreifen)
%- Mittlerweile können Chatbots auch ohne tiefgreifende Programmierkenntnisse konfiguriert werden da es zig Plattformen dafür gibt
%- Allerdings sind diese sehr allgemein und 


%Chatbot-Projekte wie Woebot oder IBM Watson Assistant nutzen die Vorteile der Geräte. Watson Assistant bietet eine Plattform um Chatbots zu  erstellen, die auch auf Gerätedaten wie Standort zugreifen. Woebot nutzt die unterschiedlichen Geräte und kann vom Nutzer via Smartphone oder Desktop PC jederzeit über verschiedene Messenger Plattformen genutzt werden.




\section{Problemstellung und Zielsetzung}

Ziel der Arbeit ist die Konzeption einer Therapiemodellierungs-sprache. Diese soll es erlauben, technisch wenig versierten Psychologen ihre Therapieideen in einer Art und Weise zu formulieren, die eine Maschine verstehen und ausführen kann. Dadurch entfällt der hohe und fehleranfällige Abstimmungsaufwand zwischen Forschern und Entwicklern. Durch den Einsatz der Therapiemodellierungs-sprache sollen MPG konforme Anwendungen mit einer conversational UI entstehen, welche eine für den Patienten vertraute,  Gesprächsähnliche Kommunikation ermöglicht. Dies erlaubt es eine stärkere persönliche Bindung zwischen App und dem Patienten herzustellen, was den Therapieerfolg unterstützen soll. 

In der Arbeit gilt es vor allem die komplexen Konfigurationsmöglichkeiten der Domäne einer digitalen Therapie funktional abzubilden. Durch eine Befragung der Anwendergruppen soll Domänenwissen erarbeitet werden und im Folgenden die Therapiemodellierungssprache und dessen grafische Repräsentation iterativ entworfen werden. Dabei gilt ein hohes Augenmerk der Usability, die gerade in der Therapie-Domäne einen hohen Stellenwert einnehmen muss.

Diese entworfene Modellierungssprache soll prototypisch umgesetzt werden und in einer kleinen Usability-Studie evaluiert werden.

%1. Ausgangssituation – Dieser erste Abschnitt stellt die Lage dar. Gleichzeitig soll er Lust darauf machen weiterzulesen. Hierfür empfiehlt es sich einen Überblick über die Ausarbeitung zu geben.
%2. Problemstellung – Eine wissenschaftliche Arbeit beschäftigt sich mit einem konkreten Problem. Je konkreter die Problemstellung, desto besser lässt sich das Thema eingrenzen. Dieser Abschnitt gibt entsprechend den Rahmen vor, in welchem die Bearbeitung erfolgen soll. Hier kann auch eine Fragestellung hinzugefügt werden, die im Fazit beantwortet wird.
%3. Zielstellung – Was ist Sinn und Zweck der Ausarbeitung? In der Einleitung sollte auf das Ende vorgegriffen werden. Was soll erreicht werden?
%4. Methodik – Was für Mittel werden auf dem Weg zum Ziel eingesetzt? Wie gehst du bei der Erstellung der Arbeit vor?

\section{Umfeld}

\section{Methodisches Vorgehen}

\section{Gliederung}

\begin{itemize}
\item Grundlagen (Kapitel 2, Definitionen, Rahmenbedingungen, Chatbots)
\item Stand der Forschung und Technik (Kapitel 3, Chatbot-Plattformen, Grafische Programmiersprachen, Auszeichnungssprachen, Experience Sampling Software, Domänenspezifische Sprachen)
\item Konzeption (Kapitel 4, Anforderungsanalyse, Ausarbeitung verschiedener Konzepte)
\item Prototypische Entwicklung der Modellierungssprache (Kapitel 5, Konzept, Umsetzung, Evaluation)
\item Ergebnisse (Kapitel 6, Zusammenfassung, Kritische Reflexion)
\item Ausblick
\end{itemize}