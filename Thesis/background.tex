%% grundlagen.tex
%% $Id: grundlagen.tex 28 2007-01-18 16:31:32Z bless $
%%

\chapter{Grundlagen}
\label{ch:Background}

\section{Definitionen}

\begin{itemize}
\item Intervention
\item Just in Time
\item Psychologischer Psychotherapeut
\end{itemize}

\section{Rahmenbedingungen von Psychotherapien}

\section{Chatbots}

%https://wirtschaftslexikon.gabler.de/definition/chatbot-54248
\begin{itemize}
\item Gabler: Chatbots oder Chatterbots sind Dialogsysteme mit natürlichsprachlichen Fähigkeiten textueller oder auditiver Art. Sie werden, oft in Kombination mit statischen oder animierten Avataren, auf Websites oder in Instant-Messaging-Systemen verwendet, wo sie die Produkte und Dienstleistungen ihrer Betreiber erklären und bewerben respektive sich um Anliegen der Interessenten und Kunden kümmern.

1. Begriff: Chatbots oder Chatterbots sind Dialogsysteme mit natürlichsprachlichen Fähigkeiten textueller oder auditiver Art. Sie werden, oft in Kombination mit statischen oder animierten Avataren, auf Websites oder in Instant-Messaging-Systemen verwendet, wo sie die Produkte und Services ihrer Betreiber erklären und bewerben respektive sich um Anliegen der Interessenten und Kunden kümmern – oder einfach dem Amüsement dienen. In sozialen Medien treten Social Bots auf, die wiederum als Chatbots fungieren können.

2. Ziele und Merkmale: Ein Chatbot untersucht die Eingaben der Benutzer und gibt Antworten und (Rück-)Fragen aus, unter Anwendung von Routinen und Regeln. In Verbindung mit Suchmaschinen, Thesauri und Ontologien sowie mithilfe der Künstlichen Intelligenz (KI) wird er zu einem breit abgestützten und einsetzbaren System. Ebenfalls unter seinen Begriff fallen Programme, die im Chat neue Gäste begrüßen, die Unterhaltung in Gang bringen sowie für die Einhaltung der Chatiquette (einer speziellen Netiquette) sorgen und beispielsweise unerwünschte Benutzer kicken.

3. Kritik und Ausblick: Chatbots waren um die Jahrtausendwende ein Hype und wurden 15 Jahre später wieder zu einem, allerdings unter neuen Voraussetzungen, wenn man an die Entwicklungen in der KI und die Überlegungen in der Ethik denkt. In der Maschinenethik werden Chatbots entwickelt, die moralisch adäquat agieren und reagieren, etwa Probleme des Gesprächspartners erkennen, eine Notfallnummer herausgeben oder ausdrücklich die Wahrheit sagen. Sie kann ebenso Lügenmaschinen als Artefakte hervorbringen, die sie dann untersucht, um wiederum Erkenntnisse in Bezug auf verlässliche und vertrauenswürdige Maschinen zu gewinnen. Die Informationsethik diskutiert die Auswirkungen des Einsatzes von Chatbots, u.a. mit Blick auf die persönliche und informationelle Autonomie. Die Wirtschaftsethik ist relevant hinsichtlich der Unterstützung und Ersetzung von Arbeitskräften

\item Verweis auf Quelle von Jürgen

\end{itemize}

