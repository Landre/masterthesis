%% entwurf.tex
%% $Id: entwurf.tex 28 2007-01-18 16:31:32Z bless $
%%
%% ==============================
\chapter{Konzeption}
\label{ch:Design}

\section{Anforderungsanalyse}

Nur Fokus auf Funktionale Anforderungen
\begin{itemize}
\item Konversationen: Arten (Fragebogen, JIT Intervention, getriggerte Intervention, On-demand Intervention, Self-Monitoring, Notfallkoffer)
\item Fragebogen (Offene und geschlossene Fragen)
\item Eingabeformate der Nutzer (Text, Likert-Skala,...)
\end{itemize}


Startseite-Inhalt		Die Startseite beinhaltet
	Logout	Es gibt ein Link zum Ausloggen
	Accounteinstellungen	Es gibt ein Link zum Aufrufen der Accounteinstellungen
	Übersicht bereits angelegter Therapie-Module	Es gibt eine Übersicht über alle bereits angelegte Therapie-Module
	Therapie-Modul hinzufügen	Es gibt eine Schaltfläche zum Hinzufügen eines Therapie-Moduls
	Therapie-Modul umbenennen	Es gibt eine Schaltfläche zum Umbenennen eines Therapie-Moduls
	Therapie-Modul entfernen	Es gibt eine Schaltfläche zum Entfernen eines Therapie-Moduls
	Therapie-Modul kopieren	Es gibt eine Schaltfläche zum Kopieren eines Therapie-Moduls
	Schaltfläche zum Exportieren eines Therapie-Moduls	Es gibt eine Schaltfläche zum Exportieren eines Therapie-Moduls
	Therapie-Modul Status	Jedes bereits angelegte Therapie-Modul erhält einen Status. Dieser Status wird auf der Übersichtsseite für jedes Therapie-Modul angezeigt.
Bearbeiten	Therapiemodul bearbeiten	Durch Klick auf ein Therapie-Modul, gelangt man auf dessen Bearbeitungsseite
Hinzufügen		Durch Klick auf die "Hinzufügen" Schaltfläche wird ein neues Therapie-Modul angelegt.
Umbenennen		Durch Klick auf die "Umbenennen" Schaltfläche kann der Nutzer das Modul umbennenen.
Kopieren		Durch Klick auf die "Kopieren" Schaltfläche kann der Nutzer das Modul kopieren.
Entfernen		Durch Klick auf die "Entfernen" Schaltfläche öffnet sich ein Dialog zur Eingabe des Wortes "DELETE" um das Entfernen des Therapie-Moduls zu bestätigen
Konversationen	Folgende Konversationsarten können mit den bereitgestellten Mitteln erstellt werden	
	Fragebogen/Datenerhebung	Es können Daten erhoben werden, z.B. via Fragebögen
	Just-In-Time Intervention	Es können Interventionen angelegt werden, die Im Anschluss zu einem Fragebogen ausgeführt wird
	Getriggerte Intervention	Diese Intervenionen werden anhand eines Triggers (Verw. Trigger) ausgelöst
	On-demand Intervention	Es können Interventionen angelegt werden, die vom Nutzer on-demand ausgeführt werden können
	Self-Monitoring	Es können Konversationen angelegt werden, die dem Nutzer zum Self-monitoring dienen
	Notfallkoffer	Es gibt eine Art Notfallkoffer. Dieser erlaubt dem Nutzer hilfreiche Übungen/Konversationen darin abzulegen um diese jederzeit aufzurufen.
Fragebogen	Fragebogen	Mit Hilfe der Eingabeformate ist es möglich, einen Fragebogen zu erstellen
	Offene Fragen	Es ist möglich offene Fragen zu stellen
	Geschlossene Fragen	Es ist möglich Geschlossene Fragen der folgenden Formate zu stellen
		Dichtonome Fragen (Ja/Nein(/Weiß nicht))
		Eingruppierungsfragen (unter 18|18-29|30-45|…)
		Ratingskalen (Likert-Skala, Visual Analog Skala)
		Summenfragen (Verteilen Sie 100 Punkte auf folgende Antworten)
		Einfachauswahl
		Mehrfachauswahl
		Ergänzungsoption
Eingabeformate	Eingabeformate der Nutzer	Text
		Likert-Skala
		Visual Analogue Skala
		Select One
		Select Many
Ausgabeformate	Ausgabeformate des Bots	Text
		Bilder
		Video
		Audio
		Link
		GIF
Plattform	Patienten	Der Patient nutzt sein Smartphone zur Teilnahme an einer Studie/einer Therapie
		Die Kommunikationsschnitstelle der Patienten wird von folgenden Plattformen unterstützt
		Android
		iOS
	Forscherplattform (Administrationsinterface)	Der Forscher nutzt einen PC zur Erstellung einer Studie/Therapie
Trigger	Folgende Methoden können zum Triggern einer Konversation eingesetzt werden	
	Therapeut	Der Therapeut schaltet eine Konversation frei, diese erscheint sofort
		Der Therapeut schaltet eine Konversation frei, diese erscheint ab einem gewissen Zeitfenster
	Zeit	Eine Konversation erscheint zu einem randomisierten Zeitpunkt innerhalb eines, vom Nutzer festgelegten, Zeitfensters [ta, te] 
		Eine Konversation erscheint zu einem, vom Forscher festgelegten, Zeitpunkt
		Eine Konversation erscheint zu einem randomisierten Zeitpunkt, innerhalb eines, vom Forscher festgelegten, Zeitfensters [ta, te]
	Liste	Konversation wird randomisiert aus einer Liste von Konversationen ausgewählt
	Ausführungsanzahl	Eine Konversation kann anhand einer Anzahl Ausführungen an einem Tag getriggert werden
	Nutzer	Ein Nutzer kann selbst eine Konversation starten 
	Konversation	Eine Konversation kann durch das Abschließen einer anderen Konversation getriggert werden
	Snooze	Eine Konversation wird erneut getriggert nach Zeitpunkt des Snoozes eines Nutzers +t (zB. 30 Minuten)
	Variable	Eine Konversation wird durch einen Wert einer Variable getriggert
		die Variable hat den aktuellen Wert x
		die Variable hatte den Wert x zum Zeitpunkt t
		die Variable wird ermittelt aus verschiedenen Variablen/Werten durch eine vorgegebene Funktion und erfüllt eine Bedingung
Nutzer	Gruppen	Ein Nuzer kann in verschiedene Gruppen eingeteilt werden (Interventionsgruppe, Kontrollgruppe)
	Pseudonymer Nutzeraccount	Der Nutzer verwendet ein Pseudonym
	Avatar	Der Nutzer kann einen Avatar wählen
App	Kontakt	Die App bietet den Nutzern die Möglichkeit den Studienleiter/Therapeuten zu kontaktieren um Probleme zu melden
	Zeitleiste	Die App bietet eine Übersicht der aktiven und kommenden Übungen in Form einer Zeitleiste
	Statistik	Die App bietet dem Nutzer die Möglichkeit Daten des Self-Monitoring in Statistiken nachzuvollziehen
	Push Notifications	Die App benachrichtigt den Nutzer sobald dieser eine Konversation zu bearbeiten hat
	Einstellungen	Die App bietet dem Nutzer die Möglichkeit Einstellungen an seinem Account vorzunehmen
Variablen	Speichern	Es können Werte unter Variablennamen abgespeichert werden
	Historie	Die Werte, die eine Variable im Verlauf eines Therapiemoduls angenommen hat, werden in einer Historie hinterlegt
	Typen	Eine Variable kann einen der folgenden Typen annehmen
		String
		Gleitkommazahl
		Ganze Zahl 
		Boolean (Wahr/Falsch)
		Image (DB-Blob)
		Video (DB-Blob)
		Audio (DB-Blob)
		Link


\section{Ausarbeitung verschiedener Konzepte}




