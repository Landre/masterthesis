\thispagestyle{empty}
\vspace*{42\baselineskip}
\hbox to \textwidth{\hrulefill}
\par
Ich versichere wahrheitsgemäß, die Arbeit selbstständig angefertigt, alle benutzten Hilfsmittel vollständig und genau angegeben und alles kenntlich gemacht zu haben, was aus Arbeiten anderer unverändert oder mit Abänderungen entnommen wurde.

Karlsruhe, den 02.09.2019

\cleardoublepage

\vspace*{1em}
\begin{center}
	\textbf{Zusammenfassung}
\end{center}
\par
Diese Masterarbeit befasst sich mit der Entwicklung von Modellierungsansätzen für digitale Therapien mittels Chatbots. 

Verschiedene Studien zeigen auf, dass Chatbots im Bereich der Psychologie und Psychotherapie nützliche Werkzeuge sein können. Allerdings ist die Umsetzung solcher Chatbots für Psychologen noch immer eine Hürde, da die vorhandenen Mittel den Anforderungen der Psychologen nicht gerecht werden. 
Aus diesem Grund wurden im Rahmen dieser Arbeit die Anforderungen an eine Plattform, zur Erstellung von digitalen Therapien, mittels Chatbots näher beleuchtet. Hierfür wurden Personas aufgestellt um die Charakteristik und Anforderungen der Zielgruppe näher zu beleuchten. Des Weiteren wurden verschiedene Technologien bewertet, die für eine Umsetzung eines Therapiemodellierungsansatzes in Frage kommen. Darüber hinaus wurden Methodiken von Studien analysiert, die bereits Therapie-Inhalte nutzen, die in Form eines Chatbots umgesetzt werden könnten, um weitere Anforderungen abzuleiten.

Basierend auf den erstellten Anforderungen, wurden verschiedene Konzepte zur Umsetzung eines Therapiemodellierungsansatzes entwickelt und prototypisch umgesetzt. Es entstanden zwei Prototypen, die jeweils unterschiedliche Ansätze implementieren. Diese wurden in einer formativen Laborstudie mit acht Probanden evaluiert. Dabei wurde das Hauptaugenmerk insbesondere auf die Unterschiede gelenkt. 

Die Ergebnisse der Studie, konnten Tendenzen verschiedener Vor- und Nachteile innerhalb der Prototypen erkennen lassen. Die Ergebnisse werden auf das Projekt \emph{TherapyBuilder} der Firma \emph{movisens GmbH}angewendet.

\cleardoublepage
\vspace*{1em}
\begin{center}
	\textbf{Abstract}
\end{center}
\par

This thesis discusses the conceptualization of approaches to modelling digital therapies by means of chat bots.

Various studies have shown that chat bots can be useful tools in the fields of psychology and psychotherapy. However, the implementation thereof still poses a challenge to professionals, since existing tools do not sufficiently meet their requirements.
For that reason, we discuss the requirements of such a digital chat bot therapy platform. First, personas were identified to describe the needs of professionals in the aforementioned fields. Then, various technology were investigated and rated relative to their suitability for creating a therapy modelling platform. Furthermore, the methodology of existing studies that were not implemented using chat bots but could, conceivably, have been done so, has been reviewed, to deduce further requirements.

Based on the resulting list of requirements, multiple concepts for a therapy platform have been created and implemented in the form of two prototypes, that each reflect different approaches to such a platform. These prototypes have been evaluated in a formative laboratory study with eight subjects, with a focus on differences between the modelling approaches.

The results indicate tendencies regarding advantages and drawbacks of the chosen approaches. The concepts developed in this thesis will be applied in the context of the \emph{TherapyBuilder} project of \emph{movisens GmbH}.

\cleardoublepage

