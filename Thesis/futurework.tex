%% zusammenf.tex
%% $Id: zusammenf.tex 4 2005-10-10 20:51:21Z bless $
%%

\chapter{Zusammenfassung und Ausblick}
\label{ch:FutureWork}
%% ==============================
%

Nachfolgend werden die Ergebnisse der Studie zusammengefasst und kritisch reflektiert. Abschließend wird ein Ausblick über den weiteren Projektverlauf und der Verwendung der Ergebnisse gegeben.

\section{Zusammenfassung}
Insgesamt wurde der \emph{TherapyBuilder}-Prototyp in der Studie besser bewertet. Auch zeigt sich eine Tendenz, dass die Konzepte die aufgestellten Hypothesen, aus Kapitel \ref{ch:evaluation}, belegt werden könnten. Das Konfigurationsprinzip ist insgesamt besser verständlich und übersichtlicher. Allerdings schneidet es in der Einstellung der Trigger marginal schlechter ab. Hier hebt sich das \emph{Konstruktionsprinzip} nur leicht durch das Baukastenprinzip hervor. Die dort angebotenen Bausteine sind nach ihrer Funktion kategorisiert und farblich kodiert. Die farbliche Kodierung unterstützt die sinnvolle Anordnung der Bausteine und bringt außerdem eine Ordnung in die entstehende Baumstruktur. Die damit einhergehende Flexibilität der Gestaltung hebt sich besonders positiv hervor. Allerdings ist die Anordnung von mehreren Elementen schnell unübersichtlich. Der zeitliche Verlauf ist schwer nachvollziehbar. Die getriggerten Konversationen sind schwer zu überblicken. Diese müssen händisch gesucht werden. Das Konfigurationsprinzip hingegen ist marginal schlechter nachvollziehbar bezüglich der Konfiguration der Trigger-Einstellungen. Außerdem benötigt die zeitliche Darstellung eine aussagekräftigere Legende und Tooltips, die einfache Erklärungen der vorhandenen Funktionen bieten. Das Konfigurationsprinzip könnte von den positiven Eigenschaften des Konstruktionsprinzips profitieren. Die Trigger-Einstellungen einer Konversation könnten sich beispielsweise durch eine Anlehnung an das Konstruktionsprinzip realisieren lassen. Statt einen Baum zu erstellen, der den gesamten Therapieablauf abbildet, könnte in den Trigger-Einstellungen einer Konversation ein kleiner Baum erstellt werden. Dieser ist entsprechend nur für diese Konversation repräsentativ. Ähnlich strikte Regeln für die Anordnung könnten hierbei helfen, diesen Baum innerhalb der Timeline abzubilden. So wäre die zeitliche Übersicht des Therapieverlaufs noch immer gegeben. Das Konstruktionsprinzip könnte von einer strengeren Anordnung profitieren. Beispielsweise durch den Einsatz von Spalten um eine zeitliche Übersicht abzubilden und die Suche nach Konversationen zu erleichtern. Auch könnte eine Suchfunktion für einzelne Konversationen hilfreich sein um längeres Suchen im Baum zu vermeiden.

Die Übersichtlichkeit und Verständlichkeit der Sprünge ist ebenfalls tendenziell besser im Vergleich zu den Sichtbarkeitsregeln. Bei letzteren ist der einhergehende Verlauf der Konversation schwer nachzuvollziehen. Die Augen-Metapher ist nicht ganz eindeutig und versteckt. Die Patienten-Eingabe hebt sich visuell kaum von der Chatbot-Ausgabe ab. Die Werkzeugpalette könnte von einer visuell besseren Trennung der Chatbot-Output und Patient-Input Elemente profitieren. Auch die Einstellung der Sichtbarkeitsregeln könnte verbessert werden. Zum einen könnte die Funktion dauerhaft sichtbar dargestellt werden. Eine kleine Vorschau des Konversationsablaufs aus Nutzersicht könnte einen besseren Überblick der Konversation und den Abhängigkeiten geben. Die Sprünge hingegen sind visuell gut nachvollziehbar. Das Einbauen der Sprünge könnte durch eine farbliche Unterscheidung vereinfacht werden. Die Verwendung und Einstellung der Sprünge, als auch der Sichtbarkeitsregeln, können durch eine Anleitung und Tooltips verbessert werden.


\section{Kritische Reflexion}
Insgesamt zeigen sich nach der Durchführung der Studie einige Tendenzen hinsichtlich des Vergleichs der betrachteten Konzepte. Allerdings können die Hypothesen auf der erhobenen Datenbasis nicht bestätigt werden. Hierfür benötigt es mehr Probanden um die Konzepte ausreichend zu vergleichen. Auch zeigt sich, dass die Studie in wenigen Punkten Anpassungen benötigt. So konnte die Übersichtlichkeit der Antwortoptionen innerhalb des Konversationsverlaufs nicht vollständig betrachtet werden, da die Probanden diese zwar in den Fragebögen bewerteten, allerdings trafen sie hinsichtlich dieser Hypothese keine Aussagen. Auf diese könnte der Studienleiter während der Aufgabenbearbeitung und innerhalb der Abschlussfragerunde intensiver eingehen und den Probanden stärker anleiten. Dennoch konnten viele Daten erhoben werden, die Stärken und Schwächen beider Systeme aufzeigen und zu einer Verbesserung der Konzepte beitragen können. Auch können die erhobenen Daten zu einer Verbesserung des üblichen \emph{movisensXS} System beitragen um die Verständlichkeit und Übersichtlichkeit zu verbessern. 

Während der Durchführung der Studie wurden außerdem Videoaufnahmen des Desktops sowie des Probanden, in Bild und Ton, angefertigt. Diese wurden zur Unterstützung der Auswertung der Studienergebnisse erhoben. Diese Daten könnten zusätzlich ausgewertet und in die Interpretation mit eingebracht werden. So könnten beispielsweise die Bearbeitungszeiten einzelner Aufgaben interessant sein. Aber auch weitere Äußerungen während der Aufgabenbearbeitung und Blickfolgen, könnten interessante Daten liefern und weitere Probleme oder positiven Aspekte aufdecken.



\section{Ausblick}
Die Ergebnisse der Studie weißen auf verschiedene, erste Verbesserungsmöglichkeiten der Ansätze hin. Zur weiteren Untersuchung der Ergebnisse, werden noch zwei weitere Probanden untersucht. Anschließend werden alle Aufnahmen transkribiert und auf weitere Aussagen untersucht. Auch werden die Zeiten der Aufgabenbearbeitung in die Evaluation mit einbezogen und anschließend die Ergebnisse neu diskutiert. Die daraus resultierenden Ergebnisse werden im Anschluss eingearbeitet und in einem ersten Programmatischen Entwurf in der Forscher-Plattform des \emph{TherapyBuilders} umgesetzt. Auch das \emph{movisensXS}-System wird anhand der Ergebnisse angepasst. Die Systeme werden erneut evaluiert. Für die Evaluation wird die, im Rahmen dieser Masterarbeit, beschriebenen Studie hinsichtlich ihrer Defizite verbessert und erneut, mit einer größeren Probandengruppe, durchgeführt. Angepasst wird die Studie hinsichtlich der Fragestellungen des Studienleiters. Die Ergebnisse der Studie werden anschließend verwendet um eine erste produktive Version der Forscher-Plattform des \emph{TherapyBuilders} zu veröffentlichen.