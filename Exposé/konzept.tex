\section{Konzept}
\label{ch:Konzept}
Ziel ist es, eine intuitive Modellierungssprache für digitale Therapien mittels Chatbots zu konzeptionieren und prototypisch zu entwickeln. Dabei soll folgende Frage beantwortet werden: Wie lassen sich die Vor- und Nachteile der betrachteten Ansätze zur Modellierung und Strukturierung von Programmen und Fragebögen so einsetzen, damit eine intuitive Programmierung von Chatbots ermöglicht wird? Diese soll für die Maschine verständlich und ausführbar sein. Hierbei soll es insbesondere Psychotherapeuten und Psychiatern, auch wenn diese über wenige Programmierkenntnissen verfügen, möglich sein, intuitiv Chatbos zu erstellen um neue Therapiemöglichkeiten zu entwickeln. Die Modellierungssprache soll den Psychologen befähigen ohne tagelange Einarbeitungszeit eine Therapie zu entwickeln. Darüber hinaus soll die Entwicklung einer Therapie mit der Modellierungssprache nicht länger dauern, als das Entwickeln einer solchen Therapie mit den bereits betrachteten Konzepten in Kapitel. Der Chatbot wird den späteren Nutzern über ein mobiles Endgerät zur Verfügung gestellt. Hierfür sollen zunächst die Anforderungen an eine intuitive Modellierungssprache für digitale Therapien mittels Chatbots in Gesprächen mit Psychotherapeuten und Psychiatern ermittelt werden. Anschließend werden verschiedene Konzepte auf dem Papier entwickelt und evaluiert. Das vielversprechendst Konzept soll im Laufe dieser Arbeit prototypisch umgesetzt und abschließend anhand einer Usability-Studie evaluiert werden. 