\section{Konzept}
\label{ch:Konzept}
Ziel ist es, ein Modellierungsansatz für digitale Therapien mittels Chatbots zu konzeptionieren und prototypisch umzusetzen. Dabei soll folgende Frage beantwortet werden: Wie lassen sich die Vor- und Nachteile der betrachteten Ansätze zur Modellierung und Strukturierung von Programmen und Fragebögen für die Therapie so einsetzen, damit eine Programmierung von Chatbots ermöglicht wird? Diese soll von einer Maschine verarbeitet und ausgeführt werden. Hierbei soll es möglich sein, auch mit wenigen Programmierkenntnissen, Chatbots zu erstellen um neue Therapiemöglichkeiten zu entwickeln. Die Modellierungssprache soll den Psychologen befähigen ohne tagelange Einarbeitungszeit eine Therapie zu entwickeln. %Darüber hinaus soll die Entwicklung einer Therapie mit der Modellierungssprache nicht länger dauern, als das Entwickeln einer solchen Therapie mit den bereits betrachteten Konzepten in Kapitel.
Der Chatbot wird den späteren Nutzern über ein mobiles Endgerät zur Verfügung gestellt. Hierfür sollen zunächst die Anforderungen an eine Modellierungssprache für digitale Therapien mittels Chatbots in Gesprächen mit Psychotherapeuten und Psychiatern ermittelt werden. Anschließend werden verschiedene Konzepte auf dem Papier entwickelt und evaluiert. Das vielversprechendst Konzept soll im Laufe dieser Arbeit in Form eines Mockups umgesetzt und abschließend anhand einer Usability-Studie evaluiert werden. 