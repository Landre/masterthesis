\section{Konzept}
\label{ch:Konzept}
Ziel ist es, eine intuitive Modellierungssprache für digitale Therapien mittels Chatbots zu konzeptionieren und prototypisch zu entwickeln. Dabei soll folgende Frage beantwortet werden: Wie lassen sich verschiedene Ansätze zur Modellierung und Strukturierung von Programmen und Fragebögen so kombinieren und erweitern, damit eine intuitive Programmierung von Chatbots ermöglicht wird? Diese soll für die Maschine verständlich und ausführbar sein. Hierbei soll es insbesondere Psychotherapeuten und Psychiatern, die über keinerlei Programmierkenntnissen verfügen, möglich sein, intuitiv Chatbos zu erstellen um neue Therapiemöglichkeiten zu entwickeln. Der Chatbot soll den späteren Nutzern über ein mobiles Endgerät zur Verfügung gestellt werden. Hierfür sollen zunächst die Anforderungen an eine intuitive Modellierungssprache für digitale Therapien mittels Chatbots in Gesprächen mit Psychotherapeuten und Psychiatern ermittelt werden. Anschließend werden verschiedene Konzepte auf dem Papier entwickelt und evaluiert. Das vielversprechendst Konzept soll im Laufe dieser Arbeit prototypisch umgesetzt und abschließend anhand einer Usability-Studie evaluiert werden. 