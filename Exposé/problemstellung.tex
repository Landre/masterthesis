%% Einleitung.tex
%% $Id: einleitung.tex 28 2007-01-18 16:31:32Z bless $
%%

\section{Problemstellung}
\label{ch:Problemstellung}
%% ==============================
% CLEARLY SHOW CONTRIBUTIONS AND LINK THEM TO SECTIONS

%- Mit welchem Bereich beschäftigen wir uns?
%- Welche Probleme treten hier auf?
%- Wie werden diese bisher gelöst?
%- Was wollen wir beitragen?

Sie geben Auskunft über das Wetter, nehmen Bestellungen entgegen oder wirken als Coach - Chatbots werden bereits vielseitig im Alltag eingesetzt. Auch die Psychologie profitiert von diesen Entwicklungen. 1966 entwickelte Joseph Weizenbaum mit \emph{ELIZA} den ersten Chatbot. \emph{ELIZA} sollte seinem menschlichen Gesprächspartner das Gefühl geben, dass dieser mit einem Psychiater über eine Chatoberfläche kommuniziert. Entwickelt wurde \emph{ELIZA} allerdings nicht mit der Absicht  Psychotherapie zugänglich zu machen, sondern um ein Modell zur maschinellen Verarbeitung von natürlicher Sprache zu implementieren. Was mit Joseph Weizenbaums \emph{ELIZA} begann, brachte mit der Entwicklung der Forschung und Technik schließlich einige Chatbots, wie \emph{Wysa}, \emph{Woebot} und \emph{Tess}, im Bereich der psychischen Gesundheit hervor. Sie stellen heutzutage verschiedene Methoden der Kognitiven Verhaltenstherapie bereit, die Nutzern helfen können deren Introspektion zu verbessern und die Methoden direkt anzuwenden. Dabei wirken sie wie ein Coach der jederzeit erreichbar ist. 

In den 1960-ern hatten nur wenige Zugang zu Computern. Durch ihre Bauweise benötigten diese nicht nur viel Platz, sie waren zu dieser Zeit auch sehr kostspielig. Die Technik hat sich allerdings über die Jahrzehnte hinweg stark verändert. Nicht nur wurden Computer erschwinglich und haben eine deutlich größere Rechenleistung, sie begleiten uns mittlerweile auch in Form eines Tablets oder Laptops als Personal Computer durch den Alltag. Seit Apple ihr erstes Smartphone \emph{Iphone} im Jahr 2007 einführte, eröffneten sich durch diese Geräte noch weitere technische Möglichkeiten. Smartphones entwickelten sich zu kleinen, handlicen Geräten die nahezu in jeder Tasche Platz finden. Außerdem beinhalten die Geräte heutzutage verschiedene Sensoren, haben Zugriff auf eine Vielzahl von Anwendungen und können sich mit dem Internet verbinden. Die Handlichkeit und Vielzahl an mitgebrachten Funktionen führte dazu, dass Smartphones im Jahre 2018 allein in Deutschland von bis zu 57 Millionen Personen genutzt wurden. 

Chatbot-Entwickler nutzen die technischen Vorteile die Smartphones und Personal Computer mit sich bringen. So begleitet \emph{Woebot} Menschen mit Depressionen oder inneren Unruhen mit Techniken aus der Kognitiven Verhaltenstherapie als Selbsthilfe durch den Alltag. Der Nutzer kann dabei auswählen, ob dieser über eine \emph{Iphone}-App, \emph{Android}-App oder via \emph{Facebook Messenger} mit \emph{Woebot} kommunizieren möchte. Letzteres ist auf jedem browserfähigen Gerät nutzbar. Eine Studie der \emph{Stanford School of Medicine} untersuchte den Einsatz von \emph{Woebot} hinsichtlich seiner Realisierbarkeit, Nutzerakzeptanz und die vorläufige Wirksamkeit des bereitgestellten Selbsthilfeprogramms. Das Ergebnis der Studie zeigte, dass \emph{Woebot} beinahe täglich von den Probanden genutzt wurde. Außerdem ließ sich bei diesen ein positiver Einfluss hinsichtlich ihrer Depressionsbewältigung und dem Umgang mit inneren Unruhen messen.

%Die Erkenntnis, dass Chatbots, wie \emph{Woebot}, einen positiven Einfluss auf die Depressionsbewältigung haben können, zeigt auf, dass Chatbots im Bereich der Psychologie 

Diese Ergebnisse zeigen auf, dass Chatbots im Bereich der Psychologie und Psychotherapie nützliche Werkzeuge sein können. Allerdings ist das Entwickeln solcher Chatbots für Psychologen noch immer eine Hürde. Zwar gibt es zahlreiche Baukästen zur Entwicklung von Chatbots die keine tiefgreifenden Programmierkenntnisse benötigen. Diese sind jedoch überwiegend auf den Bereich des Marketings ausgerichtet, weshalb sie in ihrem Funktionsumfang meist eingeschränkt sind. Baukästen die mehr Funktionalität bieten, benötigen lange Einarbeitungszeit und Expertenwissen in Bezug auf ihre Programmierung. Eine einfache und schnelle Umsetzung ist daher oft nicht möglich. Auch die Entwicklung eines eigenen Produktes birgt für Psychologen und Softwareunternehmen Probleme. So scheitert die Umsetzung unter anderem an Kommunikationshürden zwischen Entwicklern und Psychologen oder den komplexen Anforderungen, die medizinische Produkte zu erfüllen haben. 

Das Unternehmen \emph{movisens GmbH} entwickelt derzeit das Projekt \emph{TherapyBuilder} welches Psychologen sowie Psychotherapeuten die Möglichkeit bieten soll, Chatbots  für Studien sowie zur Therapiebegleitung einzusetzen. Im Rahmen dieser Masterarbeit wird für das Projekt TherapyBuilder die Modelliersungssprache \emph{TML} (Therapy Modelling Language) entwickelt. Ziel dieser \emph{TML} ist es, Psychologen die autonomie zu geben, ohne Expertenwissen, Chatbots zu erstellen um diese in Studien und Therapiebegleitend einzusetzen.



%- ca. 250 Apps im Google Play Store unter dem Suchbegriff `Psychische Gesundheit''
%- ca. 255 Apps im Google Play Store unter dem Suchbegriff `mental health''


%- Es hat sich gezeigt, dass Menschen zu Chatbots eine Bindung entwickeln
%- Diese Bindung soll zur Therapy genutzt werden (Beispielprojekte Vor- und Nachteile)
%- Movisens will hier ansetzen und einen Therapybuilder entwickeln
%- Auf Nachteil so und so gehe ich mit meiner Arbeit ein - Ziel so und so wird im Rahmen dieser Thesis bearbeitet
%- TML entwickelt die es psychologen ermöglicht ohne programmierkenntnisse Therapien anzulegen
%- diese sollen vom Patient in Form eines Chatbots bearbeitet werden