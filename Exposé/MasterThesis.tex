\documentclass{wissdoc}

% Autor: Roland Bless 1996-2009, bless <at> kit.edu
% ----------------------------------------------------------------
% Diplomarbeit - Hauptdokument
% ----------------------------------------------------------------
%%
%% $Id: diplarb.tex 53 2009-12-10 12:23:37Z bless $
%%
% wissdoc Optionen: draft, relaxed, pdf --> siehe wissdoc.cls
% ------------------------------------------------------------------
% Weitere packages: (Dokumentation dazu durch "latex <package>.dtx")
%\usepackage{bibgerm}
%\usepackage[backend=biber]{biblatex} 
\usepackage{csquotes} 
\usepackage{tabularx}
\usepackage{booktabs}
\usepackage{multirow}
%\usepackage{tocbibind}
\usepackage{siunitx}
\usepackage{xcolor}
\usepackage{textcomp}
\usepackage{listings}
\usepackage{newfloat,caption}
\usepackage{subcaption}
\usepackage{footnote}
\usepackage{rotating}
\usepackage{pgfplots}
\usepackage{pgfplotstable}
\usepackage{url}
\usepackage{boxhandler}
\usepackage{tabu}
\usepackage{amssymb}
%\usepackage{subfig}
%\usepackage{subcaption}
\usepackage{caption}
\usepackage{subcaption}
%\usepackage[plainpages=true]{hyperref}
\usepackage[space]{grffile}
%\usepackage[numbers,sort&compress]{natbib}
\usepackage[backend=bibtex8,natbib=true,hyperref=true,doi=false,url=false,sorting=none]{biblatex}
% \usepackage{varioref}
% \usepackage{verbatim}
\usepackage{float}    %z.B. \floatstyle{ruled}\restylefloat{figure}
%\usepackage[hidelinks]{hyperref}
% \usepackage{subfigure}
% \usepackage{fancybox} % für schattierte,ovale Boxen etc.
% \usepackage{tabularx} % automatische Spaltenbreite
% \usepackage{supertab} % mehrseitige Tabellen
% \usepackage[svnon,svnfoot]{svnver} % SVN Versionsinformation 
%% ---------------- end of usepackages -------------

%\svnversion{$Id: diplarb.tex 53 2009-12-10 12:23:37Z bless $} % In case that you want to include version information in the footer
%\hyphenation{if...-then...}
%% Informationen für die PDF-Datei
\pgfplotsset{compat=newest}

\hypersetup{
%%% styling of link inside pdf
  colorlinks,
  citecolor=black,
  filecolor=black,
  linkcolor=black,
  urlcolor=black,
%%%		
 pdfauthor={FirstName LastName},
 pdftitle={Title of Thesis}
 pdfsubject={Not set},
 pdfkeywords={Not set}
}
\DeclareFloatingEnvironment[fileext=frm,placement={!ht},name=Listing,within=section]{listing}

% Macros, nicht unbedingt notwendig
\input{macros}

% Print URLs not in Typewriter Font
\def\UrlFont{\rm}

\newcommand{\specialcell}[2][c]{%
  \begin{tabular}[#1]{@{}c@{}}#2\end{tabular}}

\newcommand\todo[1]{\textcolor{red}{TODO: #1}}

\newcommand\hlcode[1]{\textcolor{red}{#1}}

\newcommand\citeable[1]{\textcolor{green}{\hl{citeable: #1}}}

\newcolumntype{$}{>{\global\let\currentrowstyle\relax}}
\newcolumntype{^}{>{\currentrowstyle}}
\newcommand{\rowstyle}[1]{\gdef\currentrowstyle{#1}%
  #1\ignorespaces
}

\newif\ifcomment
%\commenttrue %# Show comments


%\newcommand{\blankpage}{% Leerseite ohne Seitennummer, nächste Seite rechts
% \clearpage{\pagestyle{empty}\cleardoublepage}
%}

\renewcommand{\thesection}{\arabic{section}}
%% Einstellungen für das gesamte Dokument

% Trennhilfen
% Wichtig! 
% Im german-paket sind zusätzlich folgende Trennhinweise enthalten:
% "- = zusätzliche Trennstelle
% "| = Vermeidung von Ligaturen und mögliche Trennung (bsp: Schaf"|fell)
% "~ = Bindestrich an dem keine Trennung erlaubt ist (bsp: bergauf und "~ab)
% "= = Bindestrich bei dem Worte vor und dahinter getrennt werden dürfen
% "" = Trennstelle ohne Erzeugung eines Trennstrichs (bsp: und/""oder)

% Trennhinweise fuer Woerter hier beschreiben
\hyphenation{
% Pro-to-koll-in-stan-zen
% Ma-na-ge-ment  Netz-werk-ele-men-ten
% Netz-werk Netz-werk-re-ser-vie-rung
% Netz-werk-adap-ter Fein-ju-stier-ung
% Da-ten-strom-spe-zi-fi-ka-tion Pa-ket-rumpf
% Kon-troll-in-stanz
}
\lstset{
    frame=single,
    breaklines=true,
		basicstyle=\scriptsize,
    %postbreak=\raisebox{0ex}[0ex][0ex]{\ensuremath{\color{red}\hookrightarrow\space}}
}

% Index-Datei öffnen
\ifnotdraft{\makeindex}
%%%%%%%%%%%%%% includeonly %%%%%%%%%%%%%%%%%%%
% Es werden nur die Teile eingebunden, die hier 
% aufgefuehrt sind!
\includeonly{%
titlepage,%
problemstellung,% Motivation, Zielsetzung, Gliederung
zielsetzung,% Grundlagen 
forschungsstand,   % Problembeschreibung (Detail) und Related Work
konzept,   % Beschreibung der Problemlösung (Konzepte, allg. Architektur, ...)
gliederung,  % Beschreibung der Umsetzung/Implementierung
zeitplan,      % Nachweis und Auswertung
}
\bibliography{Literature, Websites}
\usepgfplotslibrary{groupplots}
\usetikzlibrary{pgfplots.groupplots}
%\addbibresource{diplarb.bib}

%%%%%%%%%%%%%%%%%%%%%%%%%%%%%%%%%%%%%%%%%%%%%%
\begin{document}

%\frontmatter
\ifnotdraft{
 %% Titelseite
%% Vorlage $Id: titelseite.tex 54 2009-12-10 12:23:58Z bless $

\def\usesf{}
\let\usesf\sffamily % diese Zeile auskommentieren für normalen TeX Font

\newsavebox{\Erstgutachter}
\savebox{\Erstgutachter}{\usesf Prof.~Dr.~Michael Beigl}
\newsavebox{\Zweitgutachter}
\savebox{\Zweitgutachter}{\usesf Derzeit noch offen}

\begin{titlepage}
\setlength{\unitlength}{1pt}

\begin{picture}(0,0)(85,770)
\includegraphics[width=\paperwidth]{logos/KIT_Deckblatt}
\end{picture}

\vspace*{-39pt}\hspace*{300pt}\includegraphics[width=.27\paperwidth]{logos/TECO_KIT}

\thispagestyle{empty}

%\begin{titlepage}
%%\let\footnotesize\small \let\footnoterule\relax
\begin{center}
\hbox{}
\vfill
{\usesf
{\huge\bfseries Konzeption und Entwicklung einer intuitiven Modellierungssprache für digitale Therapien mittels Chatbots
 \par}
\vskip 1.8cm
Expos\'{e} zur Masterarbeit\\
von\\[2mm]
\vskip 1cm

{\large\bfseries Luisa Andre\\}
Studiengang Informatik (M.Sc)\\
E-Mail: luisa.andre@student.kit.edu
\vskip 1.2cm
Lehrstuhl für Pervasive Computing Systeme/TECO\\
Institut für Telematik\\
Fakultät für Informatik\\
%Universität Karlsruhe (TH)\\[2ex]
\vskip 3cm
\begin{tabular}{p{5.5cm}l}
Erstgutachter: & \usebox{\Erstgutachter} \\
Zweitgutachter: & \usebox{\Zweitgutachter} \\
Betreuerin: & PD Dr. Andrea Schankin \\
\end{tabular}
\vskip 3cm
Projektzeitraum:\qquad 01.01.2019 -- 30.06.2019
}
\end{center}
\vfill
\end{titlepage}
%% Titelseite Ende


%%% Local Variables: 
%%% mode: latex
%%% TeX-master: "diplarb"
%%% End: 

}

%% ++++++++++++++++++++++++++++++++++++++++++
%% Hauptteil
%% ++++++++++++++++++++++++++++++++++++++++++
\graphicspath{{images/}}

%\mainmatter
%\null
\pagenumbering{arabic}
%% Einleitung.tex
%% $Id: einleitung.tex 28 2007-01-18 16:31:32Z bless $
%%

\section{Problemstellung}
\label{ch:Problemstellung}
%% ==============================
% CLEARLY SHOW CONTRIBUTIONS AND LINK THEM TO SECTIONS

%- Mit welchem Bereich beschäftigen wir uns?
%- Welche Probleme treten hier auf?
%- Wie werden diese bisher gelöst?
%- Was wollen wir beitragen?

Sie geben Auskunft über das Wetter (vgl. \cite{GoogleAl38:online}), nehmen Bestellungen entgegen (vgl. \cite{KassenSc50:online}) oder wirken als Coach (vgl. \cite{Wysayour57:online}) - Chatbots werden bereits vielseitig im Alltag eingesetzt. Auch die Psychologie profitiert von diesen Entwicklungen. 1966 entwickelte Joseph Weizenbaum mit \emph{ELIZA} den ersten Chatbot. \emph{ELIZA} sollte seinem menschlichen Gesprächspartner das Gefühl geben, dass dieser mit einem Psychiater über eine Chatoberfläche kommuniziert. Entwickelt wurde \emph{ELIZA} allerdings nicht mit der Absicht  Psychotherapie zugänglich zu machen, sondern um ein Modell zur maschinellen Verarbeitung von natürlicher Sprache zu implementieren (vgl. \cite{Weizenbaum1966}). Was mit Joseph Weizenbaums \emph{ELIZA} begann, brachte mit der Entwicklung der Forschung und Technik schließlich einige Chatbots, wie \emph{Wysa} (vgl. \cite{Wysayour57:online}), \emph{Woebot} (vgl. \cite{WoebotYo93:online}) und \emph{Tess} (vgl. \cite{TessArti99:online}), im Bereich der psychischen Gesundheit hervor. Sie stellen heutzutage verschiedene Methoden der kognitiven Verhaltenstherapie bereit, die Nutzern helfen können, deren Introspektion zu verbessern. Dabei wirken sie wie ein Coach der jederzeit erreichbar ist (vgl. \cite{Fitzpatrick2017}).  

In den 1960-ern hatten nur wenige Zugang zu Computern. Durch ihre Bauweise benötigten diese nicht nur viel Platz, sie waren zu dieser Zeit auch sehr kostspielig (vgl. \cite{SWB-11524946X}). Die Technik hat sich allerdings über die Jahrzehnte hinweg stark verändert. Nicht nur wurden Computer erschwinglich und haben eine deutlich größere Rechenleistung, sie begleiten uns mittlerweile auch in Form eines Tablets oder Laptops als Personal Computer durch den Alltag. Seit Apple ihr erstes Smartphone \emph{Iphone} im Jahr 2007 einführte, eröffneten sich durch diese Geräte noch weitere technische Möglichkeiten. Smartphones entwickelten sich zu kleinen, handlichen Geräten die nahezu in jeder Tasche Platz finden (vgl. \cite{SWB-481290869}). Außerdem beinhalten die Geräte heutzutage verschiedene Sensoren, haben Zugriff auf eine Vielzahl von Anwendungen und können sich mit dem Internet verbinden (vgl. \cite{SWB-481290869}\cite{AppStore21:online}). Die Handlichkeit und Vielzahl an mitgebrachten Funktionen führte dazu, dass im Jahre 2018 allein in Deutschland 22,74 Millionen Smartphones verkauft wurden (vgl. \cite{Zukunftd37:online}). Statistiken der \emph{Bitkom Research} ermittelten, dass im Jahr 2017 78 Prozent der Deutschen ein Smartphone verwendeten (vgl. \cite{Smartpho6:online}).

Entwickler nutzen die technischen Vorteile der Smartphones und Personal Computer. So begleitet \emph{Woebot} Menschen mit Depressionen oder inneren Unruhen mit Techniken aus der kognitiven Verhaltenstherapie als Selbsthilfe durch den Alltag (vgl. \cite{Fitzpatrick2017}). Der Nutzer kann dabei auswählen, ob dieser über eine \emph{Iphone}-App, \emph{Android}-App oder via \emph{Facebook Messenger} mit \emph{Woebot} kommunizieren möchte (vgl. \cite{WoebotYo93:online}). Letzteres ist auf jedem browserfähigen Gerät nutzbar. 

%Eine Studie der \emph{Stanford School of Medicine} untersuchte den Einsatz von \emph{Woebot} hinsichtlich seiner Realisierbarkeit, Nutzerakzeptanz und die vorläufige Wirksamkeit des bereitgestellten Selbsthilfeprogramms. Das Ergebnis der Studie zeigte, dass \emph{Woebot} beinahe täglich von 31 Probanden genutzt wurde. Außerdem ließ sich bei diesen ein positiver Einfluss hinsichtlich ihrer Depressionsbewältigung und dem Umgang mit inneren Unruhen messen (vgl. \cite{Fitzpatrick2017}). 

Eine Studie der Stanford School of Medicine untersuchte den Einsatz des Chatbots \emph{Woebot} hinsichtlich seiner Realisierbarkeit, Nutzerakzeptanz und die vorläufige Wirksamkeit des bereitgestellten Selbsthilfeprogramms. Verglichen wurden dabei zwei Gruppen. Eine dieser Gruppen, bestehend aus 31 Probanden, erhielt Zugriff auf \emph{Woebot}. Die zweite Gruppe, bestehend aus 25 Probanden, erhielt Zugriff auf das Ebook \emph{Depression} des \emph{National Institute of Mental Health}. Die Studiendauer wurde auf zwei Wochen festgelegt. Nach Ablauf der Studie zeigte sich, dass die Mehrheit der \emph{Woebot}-Gruppe beinhahe täglich den Chatbot nutzte. Auch konnte bei der Nutzung des \emph{Woebots} im Vergleich zur Nutzung des Ebooks eine größere Zufriedenheit festgestellt werden. Außerdem ließ sich bei dieser Gruppe ein signifikanten, positiven Einfluss hinsichtlich ihrer Depressionsbewältigung und dem Umgang mit inneren Unruhen messen (vgl. \cite{Fitzpatrick2017}).

Eine weitere Studie testete den Einfluss eines virtuellen Akteurs auf das Nutzerverhalten innerhalb eines klinischen Interviews. In dieser Studie wurden 145 Probanden in zwei Gruppen eingeteilt. 57 dieser Probanden führten einen Dialog mit einem virtuellen Akteur, der von einem Menschen gesteuert wurde. Die restlichen 88 Probanden unterhielten sich mit einem virtuellen Akteur, der mittels künstlicher Intelligenz kommunizierte. Das jeweilige Setting der Gruppen war allen Probanden bekannt. Gemessen wurde, unter anderem anhand eines Fragebogens, die Furcht vor negativer Bewertung (FNE), das Selbstdarstellungsverhalten (IM), die Nutzbarkeit des Systems (SU) sowie die Selbsttäuschung der Probanden (SD). Die Ergebnisse zeigten auf, dass signifikante Unterschiede zwischen den Gruppen gemessen werden konnte. So wurde festgestellt, dass Probanden, die Dialoge mit der künstlichen Intelligenz führte, einen niedrigeren FNE und IM Wert aufweisen (vgl. \cite{Gratch2014}). 

%Die Erkenntnis, dass Chatbots, wie \emph{Woebot}, einen positiven Einfluss auf die Depressionsbewältigung haben können, zeigt auf, dass Chatbots im Bereich der Psychologie 

Diese Ergebnisse zeigen auf, dass Chatbots im Bereich der Psychologie und Psychotherapie nützliche Werkzeuge sein können. Allerdings ist das Entwickeln solcher Chatbots für Psychologen noch immer eine Hürde. Zwar gibt es zahlreiche Baukästen zur Entwicklung von Chatbots die keine tiefgreifenden Programmierkenntnisse benö-tigen. Diese sind jedoch überwiegend auf den Bereich des Marketings ausgerichtet, weshalb sie in ihrem Funktionsumfang meist eingeschränkt sind. Baukästen die mehr Funktionalität bieten, benötigen lange Einarbeitungszeit und Expertenwissen in Bezug auf ihre Programmierung. Eine einfache und schnelle Umsetzung ist daher oft nicht möglich. Auch die Entwicklung eines eigenen Produktes birgt für Psychologen und Softwareunternehmen Probleme. So scheitert die Umsetzung unter anderem an Kommunikationshürden zwischen Entwicklern und Psychologen. Aber auch die komplexen Anforderungen des Medizinproduktegesetztes (MPG), die medizinische Produkte für die Herstellung oder Einführung in den Europäischen Wirtschaftsraum zu erfüllen haben, stellen eine Hürde dar (vgl. \cite{MPGnicht8:online}).  

Das Unternehmen \emph{movisens GmbH} entwickelt derzeit das Projekt \emph{TherapyBuilder} welches Psychologen und Psychotherapeuten die Möglichkeit bieten soll, Chatbots  für Studien sowie zur Therapiebegleitung einzusetzen. Im Rahmen dieser Masterarbeit wird für das Projekt \emph{TherapyBuilder} ein Modellierungsansatz \emph{TMA} (Therapy Modelling Approach) entwickelt. Ziel dieses \emph{TMA} ist es, Psychologen die Autonomie zu geben, ohne Expertenwissen Chatbots zu erstellen, um diese in Studien und therapiebegleitend einzusetzen.



%- ca. 250 Apps im Google Play Store unter dem Suchbegriff `Psychische Gesundheit''
%- ca. 255 Apps im Google Play Store unter dem Suchbegriff `mental health''

 
%- Es hat sich gezeigt, dass Menschen zu Chatbots eine Bindung entwickeln
%- Diese Bindung soll zur Therapy genutzt werden (Beispielprojekte Vor- und Nachteile)
%- Movisens will hier ansetzen und einen Therapybuilder entwickeln
%- Auf Nachteil so und so gehe ich mit meiner Arbeit ein - Ziel so und so wird im Rahmen dieser Thesis bearbeitet
%- TML entwickelt die es psychologen ermöglicht ohne programmierkenntnisse Therapien anzulegen
%- diese sollen vom Patient in Form eines Chatbots bearbeitet werden  % Einleitung
%% grundlagen.tex
%% $Id: grundlagen.tex 28 2007-01-18 16:31:32Z bless $
%%

\section{Zielsetzung \& Erkenntnisinteresse}
\label{ch:Zielsetzung}
  % Grundlagen
%% analyse.tex
%% $Id: analyse.tex 28 2007-01-18 16:31:32Z bless $

\section{Stand der Technik}
\label{ch:Forschungsstand}
Recherchen haben ergeben, dass derzeit noch keine Sprache existiert, die speziell zur Modellierung von Therapien mit Chatbots entwickelt wurden. Für die Bearbeitung der Forschungsfrage werden verschiedene Technologien bewertet. Zunächst werden Plattformen beleuchtet, die eine Erstellung von Chatbots ermöglichen, ohne weitere Programmierkenntnisse zu benötigen. Nachfolgend werden diverse grafische Programmiersprachen betrachtet. Diese bilden Programmstrukturen grafisch ab und werden somit nachvollziehbarer für den Anwender. Eine weitere Kategorie bilden die Auszeichnungssprachen. Diese ermöglichen eine vereinfachte Programmierung zur Strukturierung von Texten. Abschließend werden verschiedene Ansätze im Bereich des Experience Samplings behandelt. Zwar dienen diese zur Erstellung von Studien, allerdings beinhalten sie eine Schnittmenge an Funktionalitäten, die ebenfalls für das Designen von Therapien relevant sind. 

\subsection{Chatbot-Plattformen}
Es gibt eine Vielzahl verschiedener Plattformen zur Entwicklung von Chatbots. Diese sollen insbesondere Personengruppen adressieren, die keine oder nur wenige Programmierkenntnisse besitzen. Die entsprechenden Plattformen verwenden unterschiedliche Ansätze für die Entwicklung. 

Der Konversationsfluss der Chatbots wird auf den Plattformen unter anderem als eine Art Baum, ähnlich zur bekannten Ordnerstruktur unter Windows Betriebssystemen, angelegt und dargestellt (vgl. \cite{Dialogfl40:online} \cite{KatalogI56:online}). Chatbot-Plattformen, wie \emph{ManyChat}, \emph{Converse.ai} und \emph{Chatfuel} verwenden Diagramme zur Darstellung eines Chatverlaufes (vgl. \cite{Converse15:online} \cite{WelcomeM66:online}) oder Blocksysteme mit Referenzen auf nachfolgende Blöcke (vgl. \cite{Chatfuel3:online}). Andere Plattformen nutzen keine Darstellung des Verlaufs (vgl. \cite{BotsifyC64:online}). Im Beispiel von \emph{Botsify} oder \emph{Recast.ai} werden nur Verhaltensweisen angelegt, die durch bestimmte Nutzereingaben getriggert werden. 

Auch in der Handhabung der Nutzereingaben gibt es verschiedene Ansätze. So bieten manche Plattformen die Möglichkeit Antworten für den Nutzer des Chatbots vorzugeben (vgl. \cite{Chatfuel3:online} \cite{WelcomeM66:online}). Andere hingegen verwenden natürliche Sprachverarbeitung um Phrasen einzutrainieren. Der Ersteller des Chatbots legt fest, wie der Chatbot auf die entsprechenden Phrasen reagiert. Somit kann ein Chatbot auf Synonyme oder Redewendungen eingehen. (vgl. \cite{BotsifyC64:online} \cite{Dialogfl40:online} \cite{KatalogI56:online}) Die Chatbot-Plattform \emph{Chatfuel} verwendet beide Ansätze. So können hier Antworten vordefiniert oder Phrasen festgelegt werden. (vgl. \cite{Chatfuel3:online}). 

Damit Nutzerdaten abgespeichert und verarbeitet werden können, bieten einige Plattformen Variablen an. Dort können unter anderem Nutzername sowie Aufenthaltsort des Nutzers gespeichert und weiterverwendet werden. Der Nutzer kann auf bereits definierte Variablen zurückgreifen oder eigene anlegen. (vgl. \cite{Chatfuel3:online} \cite{Converse15:online} \cite{Dialogfl40:online} \cite{KatalogI56:online} \cite{WelcomeM66:online}). 


\subsection{Grafische Programmiersprachen}
Diese Art der Programmiersprache bedient sich visueller Elemente um Programmstrukturen verständlich abzubilden. Die visuellen Elemente können auf eine bestimmte Domäne zugeschnitten sein (vgl. \cite{WasistLa94:online}) oder beschränken sich auf die Visualisierung gängiger Programmanweisungen (vgl. \cite{BlocklyG57:online}). Die grafische Programmiersprache \emph{Labview} konzentriert sich auf die Domäne System-, Steuer- und Regelungstechnik (vgl. \cite{WasistLa94:online}). Programmiert wird, indem Elemente miteinander kombiniert werden, die als Schaltzeichen aus der Elektrotechnik bekannt sind. Nach diesem Prinzip arbeiten auch die Editoren \emph{Matlab Simulink} und \emph{Choreograph}. (vgl. \cite{Choregra47:online} \cite{Simulink28:online})

Ist keine domänenspezifische grafische Programmiersprache gewünscht oder bekannt, ist es dennoch möglich ohne tiefergreifende Programmierkenntnisse Programme zu entwickeln. Ermöglicht wird dies durch Programmiersprachen, die Programmanweisungen durch Diagramme oder Blöcke visualisieren. Für Diagramme werden unter anderem Zustandsdiagramme oder eine Form von Flussdiagrammen verwendet. (vgl. \cite{SwissEdu45:online} \cite{DRAKONEd12:online} \cite{PureData15:online}) Durch diese Vorgehensweisen lassen sich Schleifen oder Bedingte Anweisungen leicht erkennen. Eine Visualisierung mit Blöcken hingegen folgt dem Steckprinzip. So können Anweisungen in Blockform nebeneinander wie untereinander angeordnet werden. Schleifen oder Bedingungen werden durch Blöcke dargestellt, die andere Blöcke beinhalten. Diese Blöcke stellen Anweisungen dar, die innerhalb dieser Schleife oder jeweiligen Bedingung ausgeführt werden. (vgl. \cite{BlocklyG18:online} \cite{NXTSoftw71:online} \cite{SnapBuil34:online} \cite{squeakla50:online}) \emph{Lego Mindstorms NXT} verbindet das Steckprinzip der Blöcke mit domänenspezifischen Elementen der Lego Mindstorms Bausätze. Insbesondere Schleifen und Bedingungen werden als eine Art Blocksystem genutzt. (vgl. \cite{NXTSoftw71:online})


\subsection{Auszeichnungssprachen}
Eine weitere Möglichkeit zur komplexen Programmierung sind die sogenannten vereinfachten Auszeichnungssprachen. Diese arbeiten mit Text der anhand einfacher Befehle formatiert und strukturiert wird. So kann anhand eines vorangehenden Symbols Text als Überschrift definiert werden. Insbesondere \emph{Markdown} verwendet Sonderzeichen um Textabschnitte zu formatieren und strukturieren. (vgl. \cite{GettingS56:online}) 

Auch \emph{YAML} nutzt Sonderzeichen, um Listen und größere Mengen von Daten zu beschreiben. (vgl. \cite{TheOffic64:online}) \emph{BBCode} hingegen verwendet einfache Anweisungen die mit eckigen Klammern eingeleitet und abgeschlossen werden. Die Anweisung selbst wird in Form eines Buchstabens angegeben. (vgl. \cite{BBCodeor24:online})

\emph{HTML} ist die bisher geläufigste Auszeichnungssprache. Diese wird zur Strukturierung von Websites benötigt. Dabei können verschiedene Bereiche Definiert und deren Inhalte Strukturiert werden. \emph{HTML} hat die Fähigkeit durch die Verwendung von Tags komplexe Inhalte, wie Texte, Bilder, Listen und Tabellen zu strukturieren und zu formatieren. Die Tags werden mit spitzen Klammern gekennzeichnet. Im Rahmen einer Studie wurde \emph{HTML} eingesetzt um Ambulante Assesment Protokolle zu erstellen, die sowohl vom Menschen lesbar als auch vom Computer ausführbar sind. (vgl. \cite{Batalas2018})



\subsection{Experience Sampling}
Psychotherapeuten und Psychologen haben die Möglichkeit anhand bestimmter Experience Sampling Software Fragebögen für mobile Geräte zu entwickeln. (vgl. \cite{OSFSabri6:online}) Hierbei werden auch Lösungen angeboten, die  Auszeichnungssprachen verwenden. Die Software \emph{Experience Sampler} verwendet die Auszeichnungssprache \emph{JSON}, aufbauend auf \emph{YAML}, um Fragen, Anzeige- sowie Eingabeformate zu definieren. (vgl. \cite{OSFSabri6:online}) \emph{MyExperience} verwendet einen ähnlichen Ansatz. Als Auszeichnungssprache zur Entwicklung der Fragebögen wird hier auf \emph{XML} zurückgegriffen. (vgl. \cite{theMyExp48:online})

Ein weiteres Projekt zur Erstellung von Experience Sampling ist \emph{Jeeves}. Fragebögen werden mit diesem Programm über eine grafische Programmiersprache definiert. Verwendet wird hauptsächlich die grafische Programmierung mit Blöcken. Über eine weitere Oberfläche werden die Eingabeformate der Antworten festgelegt. So ist es möglich Formate wie Likert Skala, Checkboxes, Radiobuttons, Ortsabfragen und weitere zu verwenden. (vgl. \cite{Rough2017})

Die Experience Sampling Software \emph{movisensXS} nutzt Diagramme zur Beschreibung des Ablaufs eines Fragebogens. Diese werden nach einem Puzzle-Prinzip angeordnet. Die Fragen selbst, sowie Formate der Antworten, werden separat angelegt und können später im Diagramm ausgewählt werden. (vgl. \cite{movisens59:online})


     % Analyse
\section{Konzept}
\label{ch:Konzept}
Ziel ist es, eine intuitive Modellierungssprache für digitale Therapien mittels Chatbots zu konzeptionieren und prototypisch zu entwickeln. Dabei soll folgende Frage beantwortet werden: Wie lassen sich verschiedene Ansätze zur Modellierung und Strukturierung von Programmen und Fragebögen so kombinieren und erweitern, damit eine intuitive Programmierung von Chatbots ermöglicht wird? Diese soll für die Maschine verständlich und ausführbar sein. Hierbei soll es insbesondere Psychotherapeuten und Psychiatern, die über keinerlei Programmierkenntnissen verfügen, möglich sein, intuitiv Chatbos zu erstellen um neue Therapiemöglichkeiten zu entwickeln. Der Chatbot soll den späteren Nutzern über ein mobiles Endgerät zur Verfügung gestellt werden. Hierfür sollen zunächst die Anforderungen an eine intuitive Modellierungssprache für digitale Therapien mittels Chatbots in Gesprächen mit Psychotherapeuten und Psychiatern ermittelt werden. Anschließend werden verschiedene Konzepte auf dem Papier entwickelt und evaluiert. Das vielversprechendst Konzept soll im Laufe dieser Arbeit prototypisch umgesetzt und abschließend anhand einer Usability-Studie evaluiert werden.      % Entwurf
%% eval.tex
%% $Id: eval.tex 5 2005-10-10 20:55:48Z bless $

%%%%%%%%%%%%%
\section{Vorläufige Gliederung}
\label{ch:Gliederung}
%%%%%%%%%%%%%
\begin{enumerate} 
\item Einleitung
	\begin{enumerate}
	\item Problemstellung und Zielsetzung
	\item Methodisches Vorgehen
	\end{enumerate}
\item Grundlagen
	\begin{enumerate}
	\item Definitionen
	\item Chatbots
	\item Rahmenbedingungen von Psychotherapien
	\end{enumerate}
\item Konzeption und Entwicklung einer intuitiven Modellierungssprache für digitale Therapien mittels Chatbots
	\begin{enumerate}
	\item Aktueller Forschungsstand
		\begin{enumerate}
		\item Chatbot-Plattformen
		\item Grafische Programmiersprachen
		\item Auszeichnungssprachen
		\item Experience Sampling Software
		\end{enumerate}
	\item Entwicklung und prototypische Umsetzung einer Modellierungssprache
		\begin{enumerate}
		\item Anforderungsanalyse
		\item Ausarbeitung verschiedener Konzepte
			\begin{enumerate}
			\item Beschreibung dieser
			\item Evaluation
			\end{enumerate}
		\item Prototypische Umsetzung der Modellierungssprache
			\begin{enumerate}
			\item Konzept
			\item Umsetzung
			\item Evaluation
			\end{enumerate}
		\item Ergebnisse
		\end{enumerate}
	\end{enumerate}
\item Fazit
	\begin{enumerate}
	\item Zusammenfassung
	\item Kritische Reflexion
	\item Ausblick
	\end{enumerate}
\end{enumerate}

	    % Implementierung
%%% entwurf.tex
%% $Id: entwurf.tex 28 2007-01-18 16:31:32Z bless $
%%
%% ==============================
\section{Zeitplan}
\label{ch:Zeitplan}

%\begin{figure}[h]
%	\begin{minipage}{0.5\textwidth}
%		\includegraphics[scale=0.5, angle=90]{images/ganttG1.png}
%	\end{minipage}
%	\begin{minipage}{0.5\textwidth}
%		\includegraphics[scale=0.5, angle=90]{images/ganttG2.png}
%	\end{minipage}		
%\end{figure}
%
%\begin{figure}[h]	
%	\begin{minipage}{1\textwidth}
%		\includegraphics[scale=0.5, angle=90]{images/ganttG3.png}
%	\end{minipage}
%\end{figure}

\begin{figure}[h]
	\includegraphics[scale=0.43]{images/ganttG1.png}
	\newline
	\includegraphics[scale=0.43]{images/ganttG2.png}
	\newline
	\includegraphics[scale=0.43]{images/ganttG3.png}	
\end{figure}





        % Evaluierung

%% ++++++++++++++++++++++++++++++++++++++++++
%% Anhang
%% ++++++++++++++++++++++++++++++++++++++++++

\appendix
%\include{anhang_a}
%\include{anhang_b}

%% ++++++++++++++++++++++++++++++++++++++++++
%% Literatur
%% ++++++++++++++++++++++++++++++++++++++++++
%  mit dem Befehl \nocite werden auch nicht 
%  zitierte Referenzen abgedruckt
\cleardoublepage
\phantomsection
\addcontentsline{toc}{chapter}{\bibname}
%%
%\nocite{*} % nur angeben, wenn auch nicht im Text zitierte Quellen 
           % erscheinen sollen
%\bibliographystyle{itmabbrv} % mit abgekürzten Vornamen der Autoren
%\bibliographystyle{gerplain} % abbrvnat unsrtnat
% spezielle Zitierstile: Labels mit vier Buchstaben und Jahreszahl
%\bibliographystyle{itmalpha}  % ausgeschriebene Vornamen der Autoren
\printbibliography
%% ++++++++++++++++++++++++++++++++++++++++++
%% Index
%% ++++++++++++++++++++++++++++++++++++++++++
\ifnotdraft{
\cleardoublepage
\phantomsection
\printindex            % Index, Stichwortverzeichnis
}
\end{document}
%% end of file

