%% analyse.tex
%% $Id: analyse.tex 28 2007-01-18 16:31:32Z bless $

\section{Stand der Technik}
\label{ch:Forschungsstand}
Recherchen haben ergeben, dass derzeit noch keine Sprache existiert, die speziell zur Modellierung von Therapien mit Chatbots entwickelt wurden. Für die Bearbeitung der Forschungsfrage werden verschiedene Technologien bewertet. Zunächst werden Plattformen beleuchtet, die eine Erstellung von Chatbots ermöglichen, ohne weitere Programmierkenntnisse zu benötigen. Nachfolgend werden diverse grafische Programmiersprachen betrachtet. Diese bilden Programmstrukturen grafisch ab und werden somit nachvollziehbarer für den Anwender. Eine weitere Kategorie bilden die Auszeichnungssprachen. Diese ermöglichen eine vereinfachte Programmierung zur Strukturierung von Texten. Abschließend werden verschiedene Ansätze im Bereich des Experience Samplings behandelt. Zwar dienen diese zur Erstellung von Studien, allerdings beinhalten sie eine Schnittmenge an Funktionalitäten, die ebenfalls für das Designen von Therapien relevant sind. 

\subsection{Chatbot-Plattformen}
Es gibt eine Vielzahl verschiedener Plattformen zur Entwicklung von Chatbots. Diese sollen insbesondere Personengruppen adressieren, die keine oder nur wenige Programmierkenntnisse besitzen. Die entsprechenden Plattformen verwenden unterschiedliche Ansätze für die Entwicklung. 

Der Konversationsfluss der Chatbots wird auf den Plattformen unter anderem als eine Art Baum, ähnlich zur bekannten Ordnerstruktur unter Windows Betriebssystemen, angelegt und dargestellt (vgl. \cite{Dialogfl40:online} \cite{KatalogI56:online}). Chatbot-Plattformen, wie \emph{ManyChat}, \emph{Converse.ai} und \emph{Chatfuel} verwenden Diagramme zur Darstellung eines Chatverlaufes (vgl. \cite{Converse15:online} \cite{WelcomeM66:online}) oder Blocksysteme mit Referenzen auf nachfolgende Blöcke (vgl. \cite{Chatfuel3:online}). Andere Plattformen nutzen keine Darstellung des Verlaufs (vgl. \cite{BotsifyC64:online}). Im Beispiel von \emph{Botsify} oder \emph{Recast.ai} werden nur Verhaltensweisen angelegt, die durch bestimmte Nutzereingaben getriggert werden. 

Auch in der Handhabung der Nutzereingaben gibt es verschiedene Ansätze. So bieten manche Plattformen die Möglichkeit Antworten für den Nutzer des Chatbots vorzugeben (vgl. \cite{Chatfuel3:online} \cite{WelcomeM66:online}). Andere hingegen verwenden natürliche Sprachverarbeitung um Phrasen einzutrainieren. Der Ersteller des Chatbots legt fest, wie der Chatbot auf die entsprechenden Phrasen reagiert. Somit kann ein Chatbot auf Synonyme oder Redewendungen eingehen. (vgl. \cite{BotsifyC64:online} \cite{Dialogfl40:online} \cite{KatalogI56:online}) Die Chatbot-Plattform \emph{Chatfuel} verwendet beide Ansätze. So können hier Antworten vordefiniert oder Phrasen festgelegt werden. (vgl. \cite{Chatfuel3:online}). 

Damit Nutzerdaten abgespeichert und verarbeitet werden können, bieten einige Plattformen Variablen an. Dort können unter anderem Nutzername sowie Aufenthaltsort des Nutzers gespeichert und weiterverwendet werden. Der Nutzer kann auf bereits definierte Variablen zurückgreifen oder eigene anlegen. (vgl. \cite{Chatfuel3:online} \cite{Converse15:online} \cite{Dialogfl40:online} \cite{KatalogI56:online} \cite{WelcomeM66:online}). 


\subsection{Grafische Programmiersprachen}
Diese Art der Programmiersprache bedient sich visueller Elemente um Programmstrukturen verständlich abzubilden. Die visuellen Elemente können auf eine bestimmte Domäne zugeschnitten sein (vgl. \cite{WasistLa94:online}) oder beschränken sich auf die Visualisierung gängiger Programmanweisungen (vgl. \cite{BlocklyG57:online}). Die grafische Programmiersprache \emph{Labview} konzentriert sich auf die Domäne System-, Steuer- und Regelungstechnik (vgl. \cite{WasistLa94:online}). Programmiert wird, indem Elemente miteinander kombiniert werden, die als Schaltzeichen aus der Elektrotechnik bekannt sind. Nach diesem Prinzip arbeiten auch die Editoren \emph{Matlab Simulink} und \emph{Choreograph}. (vgl. \cite{Choregra47:online} \cite{Simulink28:online})

Ist keine domänenspezifische grafische Programmiersprache gewünscht oder bekannt, ist es dennoch möglich ohne tiefergreifende Programmierkenntnisse Programme zu entwickeln. Ermöglicht wird dies durch Programmiersprachen, die Programmanweisungen durch Diagramme oder Blöcke visualisieren. Für Diagramme werden unter anderem Zustandsdiagramme oder eine Form von Flussdiagrammen verwendet. (vgl. \cite{SwissEdu45:online} \cite{DRAKONEd12:online} \cite{PureData15:online}) Durch diese Vorgehensweisen lassen sich Schleifen oder Bedingte Anweisungen leicht erkennen. Eine Visualisierung mit Blöcken hingegen folgt dem Steckprinzip. So können Anweisungen in Blockform nebeneinander wie untereinander angeordnet werden. Schleifen oder Bedingungen werden durch Blöcke dargestellt, die andere Blöcke beinhalten. Diese Blöcke stellen Anweisungen dar, die innerhalb dieser Schleife oder jeweiligen Bedingung ausgeführt werden. (vgl. \cite{BlocklyG18:online} \cite{NXTSoftw71:online} \cite{SnapBuil34:online} \cite{squeakla50:online}) \emph{Lego Mindstorms NXT} verbindet das Steckprinzip der Blöcke mit domänenspezifischen Elementen der Lego Mindstorms Bausätze. Insbesondere Schleifen und Bedingungen werden als eine Art Blocksystem genutzt. (vgl. \cite{NXTSoftw71:online})


\subsection{Auszeichnungssprachen}
Eine weitere Möglichkeit zur komplexen Programmierung sind die sogenannten vereinfachten Auszeichnungssprachen. Diese arbeiten mit Text der anhand einfacher Befehle formatiert und strukturiert wird. So kann anhand eines vorangehenden Symbols Text als Überschrift definiert werden. Insbesondere \emph{Markdown} verwendet Sonderzeichen um Textabschnitte zu formatieren und strukturieren. (vgl. \cite{GettingS56:online}) 

Auch \emph{YAML} nutzt Sonderzeichen, um Listen und größere Mengen von Daten zu beschreiben. (vgl. \cite{TheOffic64:online}) \emph{BBCode} hingegen verwendet einfache Anweisungen die mit eckigen Klammern eingeleitet und abgeschlossen werden. Die Anweisung selbst wird in Form eines Buchstabens angegeben. (vgl. \cite{BBCodeor24:online})

\emph{HTML} ist die bisher geläufigste Auszeichnungssprache. Diese wird zur Strukturierung von Websites benötigt. Dabei können verschiedene Bereiche Definiert und deren Inhalte Strukturiert werden. \emph{HTML} hat die Fähigkeit durch die Verwendung von Tags komplexe Inhalte, wie Texte, Bilder, Listen und Tabellen zu strukturieren und zu formatieren. Die Tags werden mit spitzen Klammern gekennzeichnet. Im Rahmen einer Studie wurde \emph{HTML} eingesetzt um Ambulante Assesment Protokolle zu erstellen, die sowohl vom Menschen lesbar als auch vom Computer ausführbar sind. (vgl. \cite{Batalas2018})



\subsection{Experience Sampling}
Psychotherapeuten und Psychologen haben die Möglichkeit anhand bestimmter Experience Sampling Software Fragebögen für mobile Geräte zu entwickeln. (vgl. \cite{OSFSabri6:online}) Hierbei werden auch Lösungen angeboten, die  Auszeichnungssprachen verwenden. Die Software \emph{Experience Sampler} verwendet die Auszeichnungssprache \emph{JSON}, aufbauend auf \emph{YAML}, um Fragen, Anzeige- sowie Eingabeformate zu definieren. (vgl. \cite{OSFSabri6:online}) \emph{MyExperience} verwendet einen ähnlichen Ansatz. Als Auszeichnungssprache zur Entwicklung der Fragebögen wird hier auf \emph{XML} zurückgegriffen. (vgl. \cite{theMyExp48:online})

Ein weiteres Projekt zur Erstellung von Experience Sampling ist \emph{Jeeves}. Fragebögen werden mit diesem Programm über eine grafische Programmiersprache definiert. Verwendet wird hauptsächlich die grafische Programmierung mit Blöcken. Über eine weitere Oberfläche werden die Eingabeformate der Antworten festgelegt. So ist es möglich Formate wie Likert Skala, Checkboxes, Radiobuttons, Ortsabfragen und weitere zu verwenden. (vgl. \cite{Rough2017})

Die Experience Sampling Software \emph{movisensXS} nutzt Diagramme zur Beschreibung des Ablaufs eines Fragebogens. Diese werden nach einem Puzzle-Prinzip angeordnet. Die Fragen selbst, sowie Formate der Antworten, werden separat angelegt und können später im Diagramm ausgewählt werden. (vgl. \cite{movisens59:online})


\subsection{Fazit}

Zwar bieten Chatbot-Plattformen bereits einige Funktionen, allerdings fokussieren diese sich vornehmlich auf Marketing, Vertrieb und Support.(vgl. \cite{Chatfuel3:online} \cite{Converse15:online} \cite{ManyChat78:online}) Dies kann die Umsetzung einer komplexen Therapie erschweren. Übliche Elemente, wie Visuelle Analogskalen und Likert-Skalen, die häufig in der Psychologie Verwendung finden, können nur schwer oder gar nicht umgesetzt werden. Diese könnten jedoch für eine Therapie von Bedeutung sein. Die Darstellungsformen der Konversationen verschiedener Chatbot-Plattformen haben unterschiedliche Vor- und Nachteile. Bäume und Diagramme bieten eine visuelle und leicht verständliche Übersicht eines Konversationsablaufs. Je größer und komplexer dieser Ablauf allerdings wird, umso unübersichtlicher wird eine Konversation. Blocksysteme bieten zusätzlich die visuelle Darstellung von Bedingungen, aber auch hier ist ein großer Konversationsablauf schnell unübersichtlich. Viele Elemente, die in Chatbot-Plattformen eingesetzt werden, könnten für eine Therapiemodellierungssprache nützlich sein, da diese leicht nachzuvollziehen und leicht in der Handhabung sind. Allerdings ist keine der bisherigen Chatbot-Plattformen derzeit geeignet um eine komplexe Therapie umzusetzen ohne lange Einarbeitungszeit oder Einschränkungen in der Gestaltung. 

Auch eine Umsetzung mit den sogenannten grafischen Programmiersprachen wäre möglich um eigene Chatbots zu entwickeln. Überwiegend gibt es diese für spezielle Domänen wie Elektrotechnik. Das Baukastenprinzip ist hier besonders interessant da es verschiedene Funktionen visuell darstellt und später in Code übersetzt. Die visuelle Darstellung ist leicht verständlich und schnell zu Erlernen. \emph{Blockly} bedient sich diesem Prinzip um verschiedene Arten der Programmanweisungen verständlicher darzustellen. Allerdings wird ein komplexeres Programm oder System auch hier schnell unübersichtlich. Auch hier gibt es noch keine grafische Programmiersprache, die zur Umsetzung einer Therapiemodellierungssprache geeignet wäre. 

Für das Beschreiben einer Konversation wäre auch die Nutzung einer Auszeichnungssprache möglich. Das Schreiben eines Konversationsfluss wirkt hier sehr natürlich, da es dem Chatten nahe kommt. Aber auch hier kann man leicht die Übersicht verlieren, da Verzweigungen in Konversationen nicht entsprechend Intuitiv dargestellt werden können, wie es beispielsweise bei Diagrammen möglich ist. Auch liefern nicht alle Auszeichnungssprachen den Umfang von Funktionen um komplexe Therapien darzustellen. Ist dies der Fall, wird das Beschreiben einer Konversation schnell unübersichtlich. Auch die Syntax und Fehlersuche wird zeitaufwändig sofern das genutzte Programm zur Beschreibung keine oder eine rudimentäre Fehlerbehandlung mit sich bringt.  

Im Bereich des Experience Samplings werden bereits verschiedene Ansätze verwendet, die eine grafische Programmierung oder das Verwenden einer Auszeichnungssprache integrieren. Hier liegt der Fokus auf der Entwicklung von Fragebögen die zu verschiedenen Zeiten getriggert werden. Sie haben nicht den Anspruch komplexe Therapien in Form von Konversationen umzusetzen. Jedoch können Fragebögen ein wichtiges Stilmittel einer Therapie darstellen. 

Aufgrund der verschiedenen Vor- und Nachteile der vorgestellten Ansätze, wäre eine Kombination verschiedener Ansätze denkbar. 










